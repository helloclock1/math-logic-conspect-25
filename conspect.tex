\documentclass[12pt]{article}
%\usepackage{lmodern}
\usepackage[hmargin=0.5in, vmargin=1in]{geometry}
\usepackage[T2A]{fontenc}
\usepackage{amsmath, amssymb, amsfonts, amsthm}
\usepackage{graphicx,url}
\usepackage[dvipsnames]{xcolor}
%Hyphenation rules
%--------------------------------------
\usepackage{hyphenat}
\hyphenation{ма-те-ма-ти-ка вос-ста-нав-ли-вать}
%--------------------------------------
\usepackage[english, russian]{babel}
\usepackage{enumerate}
\setlength{\topskip}{0mm}\setlength{\parskip}{1ex}\setlength{\parindent}{0mm}
\usepackage{enumitem}\setlist{nolistsep}

\usepackage{listings,textcomp,color}
\lstset{language=C,upquote=true,
  basicstyle=\footnotesize,commentstyle=\textit,stringstyle=\upshape,
  numbers=left,numberstyle=\footnotesize,stepnumber=1,numbersep=5pt,
  backgroundcolor=\color{white},frame=single,tabsize=2,
  showspaces=false,showstringspaces=false,showtabs=false,
  breaklines=true,breakatwhitespace=true,escapeinside=||
}

\usepackage{fancyhdr}\pagestyle{fancy}
\lhead{Обновлено: \today}\chead{{\bf Математическая логика}}\rhead{}

\usepackage{framed}
\newenvironment{quo}{
  \begin{framed}
    \begin{minipage}{0.97\textwidth}
      \setlength{\parskip}{4mm}
    }{
    \end{minipage}
\end{framed}}

\usepackage{lastpage}
\cfoot{Страница \thepage\ из \pageref{LastPage}}

% testing shit
%---
\usepackage{cmap}
\usepackage[X2,T2A]{fontenc} % X2 can be dropped
\usepackage[utf8]{inputenc}
% Bunch of math and table packages
%\usepackage{enumitem}

% list numeration modifiers
% ---
%\renewcommand{\theenumi}{\asbuk{enumi}}
%\renewcommand{\labelenumi}{\textbf{\asbuk{enumi})}}
\renewcommand{\theenumii}{\arabic{enumii}}
\renewcommand{\labelenumii}{\arabic{enumii}.}
\AddEnumerateCounter{\Asbuk}{\@Asbuk}{\CYRM}
\AddEnumerateCounter{\asbuk}{\@asbuk}{\cyrm}
% ---
\usepackage{tikz}
%---

% some useful ahh shortcuts
% ---
\let\im\rightarrow
\let\eq\equiv
\let\la\land
\let\lv\lor
\let\n\neg
\let\un\cup
\let\an\cap
\let\wot\setminus
\let\sus\subseteq
\let\m\cdot
\let\lmd\lambda
\let\nin\notin
\let\x\times
\let\sd\bigtriangleup
\let\eps\varepsilon
\let\lc\lceil
\let\rc\rceil
\let\lf\lfloor
\let\rf\rfloor
\let\rarr\rightarrow
\let\ds\displaystyle
\def\ZZ{\mathbb{Z}}
\def\II{\mathbb{I}}
\def\QQ{\mathbb{Q}}
\def\NN{\mathbb{N}}
\def\RR{\mathbb{R}}
\def\ff{\text{\sf \perp}}
\def\tt{\text{\sf \top}}
\def\lbr{&&\\\nonumber}
\def\basedlimit{\lim\limits_{n \to \infty}}
\def\ans{\tbf{Ответ:} }
\newcommand{\tr}[1]{\text{tr}\left(#1\right)}
\newcommand{\sq}[1]{\sqrt{#1}}
\newcommand{\tbf}[1]{\textbf{#1}}
\newcommand{\tit}[1]{\textit{#1}}
\newcommand{\task}[2]{
  \begin{quo}
    {\bf #1} #2
\end{quo}}
\newcommand{\mat}[1]{
  \begin{gmatrix}[p]
    #1
\end{gmatrix}}
\newcommand{\detm}[1]{
  \begin{gmatrix}[v]
    #1
\end{gmatrix}}
\newcommand{\zhopa}[1]{$\boxed{\text{#1}}$}
\newcommand{\asdf}{%
  \hspace{-\arraycolsep}%
  \strut\vrule % the `\vrule` is as high and deep as a strut
  \hspace{-\arraycolsep}%
}
\newcommand{\sgn}[1]{\text{sgn}(#1)}
\newcommand{\md}[1]{\lvert#1\rvert}
\theoremstyle{definition}
\newtheorem{definition}{Определение}[section]
\theoremstyle{plain}
\newtheorem{theorem}{Теорема}[section]
\newtheorem{lemma}{Лемма}[section]
\theoremstyle{remark}
\newtheorem*{remark}{Замечание}
\newtheorem{statement}{Утверждение}[section]
% ---

\usepackage{float}
\usepackage{caption}
\usepackage{subcaption}
\usepackage{soul}
\usepackage{mathtools}
\usepackage{gauss}
\usepackage{nicematrix}
\usepackage{breqn}
\usepackage{permute}
\usepackage{braket}
\usepackage[normalem]{ulem}
\usepackage{systeme}
\usepackage{bbold}
\usepackage{hyperref}
\hypersetup{
  colorlinks=true,
  linkcolor=blue,
  filecolor=magenta,
  urlcolor=cyan,
  pdftitle={Overleaf Example},
  pdfpagemode=FullScreen,
}
\usepackage{forest}

\usepackage{etoolbox}
\makeatletter
\patchcmd\g@matrix
{\vbox\bgroup}
{\vbox\bgroup\normalbaselines}% restore the standard baselineskip
{}{}
\makeatother

\begin{document}
Снимаю с себя всю ответственность за нули на коллоквиуме, полученные
из-за прочтения фактов с этого конспекта. По всем неточностям и предложениям:
\href{https://t.me/helloclock}{@helloclock}.
\tableofcontents
\pagebreak

\section{Лекция 1 (введение)}

\begin{definition}
  \textit{Алфавит} $\Sigma = \set{(, ), \land, \lor, \n, \im} \un
  \mathrm{Prop} \un \set{\mathrm{True}, \mathrm{False}}$, где
  $\mathrm{Prop}$ --- множество пропозициональных переменных. В курсе
  $\mathrm{Prop} = \set{p, \dots, z, p_1, \dots, z_1, \dots}$.
\end{definition}

\begin{definition}
  \textit{Формула} --- последовательность символов из алфавита,
  определяемая по индукции:
  \begin{enumerate}
    \item T и F --- формулы;

    \item $p \in \mathrm{Prop}$ --- формула;

    \item $A$ --- формула $\implies \n A$ --- формула;

    \item $A, B$ --- формулы $\implies (A \land B), (A \lor B), (A
      \im B)$ --- формулы.
  \end{enumerate}
\end{definition}

Формулы удобно представлять в виде дерева, например для формулы
\begin{displaymath}
  ((p \land q) \im (\n r \lor p))
\end{displaymath}
дерево выглядит следующим образом:
\begin{figure}[H]
  \centering
  \begin{forest}
    [$\rightarrow$
      [$\land$
        [$p$]
        [$q$]
      ]
      [$\lor$
        [$\neg$
          [$r$]
        ]
        [$p$]
      ]
    ]
  \end{forest}
\end{figure}

\begin{definition}
  Пусть $A$ --- последовательность символов в алфавите $\Sigma$:
  \begin{displaymath}
    A = a_1 \dots a_n\ (a_j \in \Sigma).
  \end{displaymath}
  Тогда $B$ --- \textit{префикс} ($B \sqsubseteq A$), если $B = a_1
  \dots a_k$ ($k \leqslant n$).
\end{definition}

\begin{lemma}
  Если $A$ --- корректная формула и $A'$ --- её префикс, то $A'$ ---
  не корректная формула.
\end{lemma}

\begin{lemma}[Об однозначности разбора]
  Если $A$ --- корректно построена, то верно ровно одно из следующего:
  \begin{enumerate}
    \item $A \in \mathrm{Prop}$

    \item $A \in \set{\top, \perp}$

    \item $\exists! B \colon A = \n B$

    \item $\exists! B, C \colon A = (B * C)$, где $* \in \set{\la, \lv, \im}$
  \end{enumerate}
\end{lemma}

\pagebreak

\section{Лекция 2 (модели, ДНФ)}

\subsection{Модели}

\begin{definition}
  \textit{Модель} --- функция $\mathrm{Var} \to \mathbb{B} = \set{0,
  1}$. Но т.к. Var бесконечно, в программах будем считать моделю
  любую функцию из конечного множества переменных. В таком случае
  некоторым переменным значение не приписывается.
\end{definition}

Модель задаёт интерпретацию истинности всех формул.

Для формулы $A$ \textit{истинность} формулы в модели $M$ задаётся по
индукции и обозначается $M(A)$:
\begin{itemize}
  \item $M(\top) = 1$, $M(\perp) = 0$;

  \item На переменных уже задано;

  \item $M(\n B) = 1 - M(B)$, если $A = \n B$;

  \item $M(B_1 \odot B_2) = M(B_1) \odot M(B_2)$, если $A = (B_1
    \odot B_2)$, $\odot \in \set{\lor, \land, \im}$.
\end{itemize}

\begin{definition}
  \textit{Булевая функция} --- отображение $\mathbb{B}^k \to
  \mathbb{B}$, задаёт булеву функцию данной формулу для фиксированного
  порядка переменных.
\end{definition}

\subsection{Виды формул}

\begin{definition}
  \textit{Тавтология} --- формула, истинная во всех моделях

  \textbf{Пример:} $p \im p$, $p \lor \n p$
\end{definition}

\begin{definition}
  \textit{Тождественно ложная/противоречивая} формула --- формула,
  ложная во всех моделях.

  \textbf{Пример:} $p \land \n p$
\end{definition}

\begin{definition}
  \textit{Выполнимая} формула --- формула, истинная хотя бы в одной модели.

  \textbf{Пример:} $p \land q$\\
\end{definition}

Пример:

\task{}{Если в пробах с Европы (спутник Юпитера) обнаружены бактерии,
  то на Европе есть жизнь или бактерии были занесены с Земли. Если
  бактерии были занесены с Земли, то на Земле есть похожие бактерии. В
  пробах с Европы обнаружены бактерии, похожие на Земные, следовательно
на Европе нет жизни.}
\begin{itemize}
  \item $p$ --- "В пробах с Европы обнаружены бактерии"

  \item $q$ --- "На Европе есть жизнь"

  \item $r$ --- "Бактерии с Земли"

  \item $s$ --- "Бактерии похожи не Земные"
\end{itemize}

Утверждение можно записать следующей формулой:

\[
  \left((p \im q \lor r) \land (r \im s) \land s\right) \im \n q
\]

Если она тавтологична, то утверждение верно, иначе --- нет.

Чтобы проверить на тавтологичность, надо проверить, есть ли набор
переменных, для которого формула ложна, тогда она будет не
тавтологична. Для этого первая скобка должна быть истинной, а вторая
--- ложной. Отсюда $q = \top$, $(p \im q \lor r) = \top$, $(r \im s)
= \top$, $s = \top$. Из имеющегося получаем $p = q = r = s = \top$.
Для этого набора переменных утверждение ложно, т.е. оно не
тавтологично, а значит --- не истинно во всех моделях. \qed

\subsection{ДНФ}

\begin{definition}
  \textit{Литерал} --- переменная или её отрицание.
\end{definition}

\begin{definition}
  \textit{Элементарная конъюнкция/конъюнкт} --- конъюнкция литералов.
\end{definition}

\begin{definition}
  \textit{Дизъюнктивная нормальная форма (ДНФ)} --- дизъюнкция конъюнктов.
\end{definition}

\begin{definition}
  \textit{Совершенная дизъюнктивная нормальная форма (СДНФ)}:
  \begin{itemize}
    \item Определена для фиксированного множества переменных;

    \item ДНФ, в которой в каждом конъюнкте участвуют все переменные из
      множества и только один раз.
  \end{itemize}
\end{definition}

Построение СДНФ:
\begin{itemize}
  \item Можно построить по таблице истинности при условии, что в ней
    есть хотя бы одна 1;

  \item Каждая строка преобразуется в элементарную конъюнкцию,
    которая истинна только на данном наборе переменных и ложна на
    всех остальных;

  \item Итоговая формула --- дизъюнкция построенных конъюнктов.
\end{itemize}

\begin{theorem}[Теорема о функциональной полноте]
  Для любой булевой функции существует булева формула, задающая эту функцию.
\end{theorem}

\subsection{Алгоритмическая сложность}
\subsubsection{Задача о нахождении ДНФ по таблице истинности}
\begin{itemize}
  \item Прямой алгоритм перебирает строки таблицы истинности;

  \item Полиномиальная сложность по размеру таблицы истинности.
\end{itemize}

\subsubsection{Задача о проверке тавтологичности}
\begin{itemize}
  \item Проверяет, истинна ли формула во всех моделях;

  \item Связана с проблемой SAT (проблема выполнимости);

  \item NP-полнота: нахождение эффективного алгоритма неизвестно;

  \item Есть очень хорошие SAT-решатели, которые применяют различные
    эвристики и быстро работают на формулах, которые появляются в
    реальных задачах.
\end{itemize}

\pagebreak

\section{Лекция 3 (полнота и максимальность)}
\begin{definition}
  \textit{Арность} операции (функции) --- количество аргументов.
\end{definition}

Арность может быть равной 0 --- это константы.

Арность:
\begin{itemize}
  \item 0 --- операций всего 2 ($\top, \perp$)

  \item 1 ---
    \begin{tabular}{c||c|c|c|c}
      $x$ & $\perp$ & $x$ & $\n x$ & $\top$\\\hline
      0 & 0 & 0 & 1 & 1\\\hline
      1 & 0 & 1 & 0 & 1
    \end{tabular}

  \item 2 ---
    \begin{tabular}{c|c||c|c|c|c}
      $x$ & $y$ & $f_1$ & $f_2$ & $f_3$ & $f_4$\\\hline
      0 & 0 & 0 & 0 & 0 & \dots\\\hline
      0 & 1 & 0 & 0 & 0 & \dots\\\hline
      1 & 0 & 0 & 0 & 1 & \dots\\\hline
      1 & 1 & 0 & 1 & 0 & \dots\\
    \end{tabular}
\end{itemize}
\begin{statement}
  Штрих Шеффера, имеющий следующую таблицу истинности:
  \begin{figure}[H]
    \centering
    \begin{tabular}{c|c|c}
      $x$ & $y$ & $x \uparrow y$\\\hline
      0 & 0 & 1\\\hline
      0 & 1 & 1\\\hline
      1 & 0 & 1\\\hline
      1 & 1 & 0
    \end{tabular}
  \end{figure}
  --- полная операция, для системы $\set{ \uparrow }$ верна
  теорема о функциональной полноте:
  \begin{itemize}
    \item $x \uparrow x = \n x$

    \item $x \uparrow y = \n(x \land y)$

    \item $\n (x \uparrow y) = \n \n (x \land y) = x \land y$

    \item $x \lor y = \n (\n x \land \n y)$
  \end{itemize}
\end{statement}
\begin{statement}
  Стрелка Пирса, имеющая следующую таблицу истинности:
  \begin{figure}[H]
    \centering
    \begin{tabular}{c|c|c}
      $x$ & $y$ & $x \downarrow y$\\\hline
      0 & 0 & 1\\\hline
      0 & 1 & 0\\\hline
      1 & 0 & 0\\\hline
      1 & 1 & 0
    \end{tabular}
  \end{figure}
  --- также
  полная операция и для неё выполнена теорема о функциональной полноте.
\end{statement}
\begin{definition}
  $G$ --- множество булевых функций, тогда $[G]$ ---
  \textit{замыкание} множества $G$, т.е. все булевы функции, которые
  можно выразить формулами, использующими операции из $G$.

  Эквивалентно, $[G]$ --- минимальное множество булевых функций,
  которое удовлетворяет следующим свойствам:
  \begin{enumerate}
    \item $G \subset [G]$;

    \item
      \begin{enumerate}
        \item $[G]$ замкнуто относительно композиции;

        \item $\forall f_1, \ldots, f_n \in [G] \land g(x_1, \ldots, x_n)
          \in G \hookrightarrow g(f_1(x_1, \ldots, x_{m_1}), \ldots,
          f_n(x_1, \ldots, x_{m_n})) \in [G]$
      \end{enumerate}

    \item $[G]$ содержит все тождественные проекции, т.е. $\forall n
      \in \NN, i < n \colon p_i^n(x_1, \ldots, x_n) \eq x_i$
  \end{enumerate}
\end{definition}
\begin{definition}
  Множество (класс) функций $G$ называется \textit{замкнутым}, если $[G] = G$.
\end{definition}
\begin{definition}
  Класс функций $G$ называется \textit{полным}, если $[G] = F_n$, где
  $F_n$ --- множество всех булевых функций $n$ переменных.
\end{definition}
\begin{definition}
  Класс $G$ называется \textit{максимальным}, если это замкнутый
  собственный ($\neq F_n$) класс, такой, что $\forall f \in F_m \wot
  G \hookrightarrow G \un \set{f}$ --- полный класс. Эквивалентно,
  $[G \un \set{f}] = F_n$.
\end{definition}
\begin{definition}
  Класс функций $H$ \textit{неполный}, если $\exists\, G$ ---
  замкнутый, такой, что $G \neq F_n \land H \subset G$.
\end{definition}
\begin{lemma}
  Свойства замыкания:
  \begin{enumerate}
    \item $G \subset [G]$

    \item $G \subset H \implies [G] \subset [H]$

    \item $[G] = [[G]]$
      \begin{proof}
        \begin{enumerate}
          \item Очевидно

          \item Пусть $f \in [G]$, тогда она получена по 1 и 2
            свойствам замыкания из функций в $G$. Тогда очевидно, что
            $f \in [H]$.

          \item
            $\subset$ --- следует из первых двух пунктов

            $\supset$ --- докажем по индукции. Пусть $f \in [[G]]$. Тогда
            пункты 1 и 3 из определения тривиальны. Проверим 2.

            Факт: проекции позволяют увеличивать число переменных
            некоторыми мнимыми. Например, $g(p_1^3(x_1, x_2, x_3),
            p_2^3(x_1, x_2, x_3))$.

            Пусть $f(x_1, \ldots, x_n) = g(f_1(x_1, \ldots, x_n), \ldots,
            f_m(x_1, \ldots, x_n))$. По предположению индукции $f_i \in
            [G]\ \forall\, i$. Таким образом, $f \in [G]$ как композиция.
        \end{enumerate}
      \end{proof}
  \end{enumerate}
\end{lemma}
\begin{lemma}
  $H$ не является полной $\iff$ $\exists\, G$ --- максимальная и $H
  \subseteq G$.
  \begin{proof}
    $\impliedby$ --- очевидно \copyright

    $\implies$ если $H$ --- неполная, тогда $[H]$ --- замкнута и неполна.
    \begin{enumerate}
      \item Случай 1: $[H]$ --- максимальна, тогда всё хорошо.

      \item Случай 2: $[H]$ --- не максимальна $\implies \exists\, f
        \in F_m \wot [H] \colon [[H] \un \set{f}] \neq F_m$. Тогда
        пусть $H := [H] \un \set{f}$ и вернёмся в начало.
        Теоретически, этот процесс может не сойтись. Сходимость
        доказывается леммой Цорна и трансфинитной индукцией или из
        теоремы Поста.
    \end{enumerate}
  \end{proof}
\end{lemma}
\begin{theorem}[Поста]
  $T_0 = \set{f \mid f(0, \ldots, 0) = 0}$
  \begin{lemma}
    $T_0$ --- максимальный замкнутый класс.
    \begin{proof}
      \begin{itemize}
        \item Замкнутость:
          \begin{displaymath}
            \underbrace{g}_{\in T_0}(\underbrace{f_1(\vec{0})}_{\in
              T_0}, \ldots,
            \underbrace{f_n(\vec{0})}_{\in T_0}) = 0
          \end{displaymath}

        \item Максимальность:
          Пусть $h \nin T_0$, тогда $h(0, \ldots, 0) = 1$. Получаем два случая:
          \begin{itemize}
            \item $h(1, \ldots, 1) = 1 \implies h(x, \ldots, x) \eq 1$

              Возьмём полную систему $\set{\oplus, \la, 1}$. 1 уже имеем, а для
              каждой из остальных функций множества принадлежность к
              классу $T_0$ очевидна.

            \item $h(1, \ldots, 1) = 0 \implies h(x, \ldots, x) \eq \n x$

              Заметим, что также имеем в $T_0$ конъюнкцию (т.к. $0
              \land 0 \eq 0 \implies \land \in T_0$). Тогда с помощью
              неё и отрицания выразим все остальные операции.
          \end{itemize}

      \end{itemize}
    \end{proof}
  \end{lemma}

  \begin{definition}
    $T_1 = \set{f \mid f(1, \ldots, 1) = 1}$. Лемма и её
    доказательство аналогичны $T_0$, за исключением того, что если
    $h(x, \dots, x) = 0$, то берём полную систему $\set{\im, \perp}$.
  \end{definition}
\end{theorem}

\pagebreak

\section{Лекция 4 (теорема Поста)}

\subsection{Продолжение про классы функций}

\begin{definition}
  $M$ --- класс монотонных функций, содержащий функции, неубыващие по
  каждому аргументу. Т.е., если для наборов аргументов
  \begin{displaymath}
    f(x_1, \dots, x_n), f(y_1, \dots, y_n)
  \end{displaymath}
  верно
  \begin{displaymath}
    x_1 \leqslant y_1, \dots, x_n \leqslant y_n,
  \end{displaymath}
  то выполняется
  \begin{displaymath}
    f(\vec{x}) \leqslant f(\vec{y}).
  \end{displaymath}
  Пример монотонной функции: $x_1 \la x_2$.

  Пример немонотонной функции: $x_1 \im x_2$.
\end{definition}

\begin{lemma}
  $M$ является максимальным замкнутым классом.

  \begin{proof}
    Докажем замкнутость. Возьмём набор функций
    \begin{displaymath}
      g(x_1, \dots, x_n), f_1(y_1, \dots, y_m), \dots, f_n(y_1,
      \dots, y_m) \in M.
    \end{displaymath}
    Рассмотрим $g(f_1(\vec{y}), \dots, f_n(\vec{y}))$, и возьмём
    $\vec{y'}$ такой, что $\forall i \colon y_i \leqslant y'_i$.
    Тогда $\forall i \colon f_i(\vec{y}) \leqslant f_i(\vec{y'})$,
    откуда получаем $g(f_1(\vec{y}), \dots, f_n(\vec{y})) \leqslant
    g(f_1(\vec{y'}), \dots, f_n(\vec{y'}))$.

    Теперь докажем максимальность. Возьмём $h \nin M$. Заметим, что
    $0, 1 \in M$. Тогда $\exists x_1 \leqslant y_1, \dots, x_m
    \leqslant y_m \colon h(\vec{x}) > h(\vec{y})$, т.е. $h(\vec{x}) =
    1$ и $h(\vec{y}) = 0$. Пусть $\vec{x} \preccurlyeq
    \vec{y} \iff x_1 \leqslant y_1 \la \dots \la x_n \leqslant y_n$.

    \begin{lemma}[Вспомогательная лемма]
      Пусть $\vec{x} \preccurlyeq \vec{y}$. Тогда $\exists \vec{x}^1,
      \dots, \vec{x}^k \colon \vec{x}^1 = \vec{x} \la \vec{x}^k =
      \vec{y}$ и $\vec{x}^i$ отличается от $\vec{x}^{i+1}$ в одной
      координате и $\vec{x} = \vec{x}^1 \preccurlyeq \vec{x}^2
      \preccurlyeq \dots \preccurlyeq \vec{x}^k = \vec{y}$.
      \begin{proof}
        Доказательства не было, интуитивно --- путь в $n$-мерном
        булевом кубе от вершины $\vec{x}$ до вершины $\vec{y}$.
      \end{proof}
    \end{lemma}
    Посмотрим значение функции $h$ на точках $\vec{x}^i$. Тогда
    $h(\vec{x}^1) = 1$, $h(\vec{x}^k) = 0$. Понятно, что тогда
    $\exists i \colon h(\vec{x}^i) = 1 \la h(\vec{x}^{i+1}) = 0$.
    Пусть у $\vec{x}^i$ на $j$-ой позиции стоит 0, а у
    $\vec{x}^{i+1}$ --- 1. Получаем
    \[
      h(\dots \underbrace{0}_j \dots) = 1, h(\dots \underbrace{1}_j
      \dots) = 0 \implies h(\dots \underbrace{p}_j \dots) = \n p
    \]
    Таким образом получили отрицание. С другой стороны, конъюнкция
    также монотонна, а с помощью этих двух функций уже выразим все остальные.
  \end{proof}
\end{lemma}

\begin{definition}
  $S$ --- класс самодвойственных функций, т.е функций, удовлетворяющих условию
  \begin{displaymath}
    f(\overline{x_1}, \dots, \overline{x_n}) = \overline{f}(x_1, \dots, x_n)
  \end{displaymath}
  Пример несамодвойственной функции: $x \la y$.

  Пример самодвойственной функции: $\n x$, $x \oplus y \oplus z$.
\end{definition}
\begin{lemma}
  $S$ является максимальным замкнутым классом.
  \begin{proof}
    Докажем замкнутость. Пусть
    \begin{displaymath}
      f_1, \dots, f_n \colon \mathbb{B}^m \to \mathbb{B}; g \colon
      \mathbb{B}^n \to \mathbb{B} \in S
    \end{displaymath}
    Рассмотрим $g(f_1(\overline{y_1}, \dots, \overline{y_m}), \dots,
    f_n(\overline{y_1}, \dots, \overline{y_m})) =
    g(\overline{f_1(\vec{y})}, \dots, \overline{f_n(\vec{y})}) =
    \overline{g(f_1(\vec{y}), \dots, f_n(\vec{y}))}$.

    Докажем максимальность. Пусть $h \nin S$, тогда $\exists x_1,
    \dots, x_n \colon h(\overline{x_1}, \dots, \overline{x_n}) =
    h(x_1, \dots, x_n)$.

    Обозначим $p^0 = \n p$, $p^1 = p$. Тогда $h(p^{x_1}, \dots,
    p^{x_n}) = h(\overline{p^{x_1}}, \dots, \overline{p^{x_n}})$. Заметим, что
    \begin{displaymath}
      h(0^{x_1}, \dots, 0^{x_n}) = h(\overline{x_1}, \dots,
      \overline{x_n}),\ h(1^{x_1}, \dots, 1^{x_n}) = h(x_1, \dots, x_n)
    \end{displaymath}
    Пусть $g(p) := h(p^{x_1}, \dots, p^{x_n})$, тогда возможно два случая:
    \begin{enumerate}
      \item $g(p) \eq 1$, тогда получаем $\n 1 = 0$

      \item $g(p) \eq 0$, тогда получаем $\n 0 = 1$
    \end{enumerate}
    То есть имеем константы 0 и 1.

    Рассмотрим $V(x_1, x_2, x_3) = x_1 \la x_2 \oplus x_2 \la x_3
    \oplus x_1 \la x_3$. Тогда в поле $\mathbb{F}_2$ получаем:
    \begin{align*}
      (x_1 + 1)(x_2 + 1) + (x_2 + 1)(x_3 + 1) + (x_1 + 1)(x_3 + 1) &=
      \dots =\\&=
      x_1x_2 + x_2x_3 + x_1x_3 + 1 =\\&= \overline{x_1x_2 + x_2x_3 + x_1x_3}
    \end{align*}
    Т.е. $V \in S$. Заметим, что $V(x_1, x_2, 0) = x_1 \land x_2$, а
    также, что $\n \in S$. В итоге получаем полную систему $\set{\n, V, 0}$.
  \end{proof}
\end{lemma}

\begin{definition}
  \textit{Многочлен Жегалкина} --- многочлен над полем
  $\mathbb{F}_2$. Эквивалентно можно считать, что это формула с
  операциями $\la, \oplus, 1$, представляющая из себя сумму $\oplus$
  элементарных конъюнкций (\textit{одночленов Жегалкина}) и, возможно, 1.
\end{definition}

\begin{lemma}
  Все булевы функции однозначно (с точностью до перестановки
  слагаемых и сомножителей) представляются в виде многочлена Жегалкина.
  \begin{proof}
    Всего одночленов Жегалкина от $n$ переменных $2^n$. Всего
    многочленов Жегалкина, соответственно, $2^{2^n}$. Булевых функций
    $\mathbb{B}^n \to \mathbb{B}$ тоже $2^{2^n}$. А т.к. каждая
    булева функция представима в виде многочлена Жегалкина (т.е. есть
    сюръекция) и их число одинаково, то имеем и биекцию между ними.
  \end{proof}
\end{lemma}

\begin{definition}
  \textit{Степень} многочлена Жегалкина равна количеству переменных в
  нём. \textit{Линейными} называются многочлены, в которых все
  одночлены степени не больше 1.
\end{definition}

\begin{definition}
  $L$ --- класс функций, эквивалентных некоторому линейному
  многочлену Жегалкина.
\end{definition}

\begin{lemma}
  $L$ является максимальным замкнутым классом.
  \begin{proof}
    Пусть $g \nin L$; $0, 1 \in L$ и определим $g(x_1, \dots, x_n) =
    x_1x_2\dots x_m + \dots$. Рассмотрим $g(x_1, x_2, 1, \dots, 1)$:
    \[
      g(x_1, x_2, 1, \dots, 1) =
      \begin{cases}
        x_1x_2\\
        x_1x_2 + 1\\
        x_1x_2 + x_1\\
        x_1x_2 + x_2\\
        x_1x_2 + x_1 + 1\\
        x_1x_2 + x_2 + 1\\
        x_1x_2 + x_1 + x_2\\
        x_1x_2 + x_1 + x_2 + 1
      \end{cases}
    \]
    $x_1, x_2 \in L$, тогда в каждом случае можем добавить нужное
    число раз $x_1, x_2, 1$ к выражению, чтобы получить $x_1x_2 \eq
    x_1 \land x_2$. Получаем полную систему $\set{ \land, \n }$.
  \end{proof}
\end{lemma}

\begin{theorem}[Поста]
  Множество булевых функций $H$ не является полным тогда и только
  тогда, когда оно содержится в одном из классов $T_0, T_1, M, S, L$.
  \begin{proof}
    План доказательства:
    \begin{itemize}
      \item $\impliedby$: Если $H$ содержится в каком-то собственном
        замкнутом классе, то он не полон;

      \item $\implies$: Покажем обратное. Пусть $H$ не лежит целиком
        ни в одном из перечисленных классов:
        \begin{itemize}
          \item Возьмём функцию $f_0 \in H \colon f_0 \nin T_0$.
            Тогда $f_0(x, \dots, x)$ либо равна 1, либо $\n x$.

          \item Возьмём функцию $f_1 \in H \colon f_1 \nin T_1$.
            Тогда $f_1(x, \dots, x)$ либо равна 0, либо $\n x$.

          \item Если есть $\n x$: используем несамодвойственную
            функцию $f_S$ и получим одну из констант.

          \item Если есть 0 и 1, тогда используем немонотонную
            функцию $f_M$ и получим отрицание $\n x$.

          \item У нас есть 0, 1 и $\n$. Используя нелинейную функцию
            $f_L$ можем получить $\la$
        \end{itemize}
    \end{itemize}
  \end{proof}
\end{theorem}

\subsection{Замены}

\begin{definition}
  Замену переменной $p$ на формулу $\psi$ в формуле $\varphi$
  обозначается $\varphi[p/\psi]$.
\end{definition}

\begin{theorem}
  Пусть формулы $\psi_1$ и $\psi_2$ имеют одинаковые таблицы
  истинности ($\psi_1 \eq \psi_2$), тогда для любой формулы $\varphi$
  \begin{displaymath}
    \varphi[p/\psi_1] \eq \varphi[p/\psi_2]
  \end{displaymath}
  \begin{proof}
    По индукции:

    \textbf{База}: для $\varphi = \perp / \top$ очевидно, как и для
    $\varphi = q \neq p$. Для $\varphi = p$ получаем $\varphi[p /
    \psi_1] = \psi_1$ и $\varphi[p / \psi_2] = \psi_2$, а они равны.

    \textbf{Шаг}: $\varphi = \varphi_1 \la \varphi_2$

    $\varphi[p / \psi_1] = \varphi_1[p / \psi_1] \la \varphi_2[p /
    \psi_1]$, аналогично для $\psi_2$. Тогда, по предположению
    индукции, $\varphi_1[p / \psi_1] \eq \varphi_1[p / \psi_2]$ и
    $\varphi_2[p / \psi_1] \eq \varphi_2[p / \psi_2]$, а объединение
    этих формул не влияет на эквивалентность.
  \end{proof}
\end{theorem}

\pagebreak

\section{Лекция 5 (выводы)}

\subsection{Правила вывода}

\begin{definition}
  \textit{Правилом вывода} будем называть пару, состоящую из
  множества формул $\Gamma$ и одной формулы $\varphi$. При этом
  $\Gamma$ может быть пустым. $\Gamma$ будем называть множеством
  \textit{посылок}, а формулу $\varphi$ \textit{заключением}. Правила
  вывода обычно записывают так:
  \begin{displaymath}
    \frac{\Gamma}{\varphi}\ \text{или}\ \frac{\psi_1, \dots, \psi_n}{\varphi}
  \end{displaymath}
  Теоретически можно рассматривать правила, в которых $\Gamma$
  бесконечно, такие правила называются \textit{инфинитарными}, но мы
  так делать не будем, у нас всё конечно.

  Пусть $\Gamma$ --- множество формул (необязательно конечное), и
  $\varphi$ --- формула. Будем говорить, что из $\Gamma$
  \textit{логически следует} $\varphi$, если в любой модели $M$, в
  которой истинны все формулы из $\Gamma$ истинна и формула $\varphi$
  (обозначение: $\Gamma \models \varphi$). Правило
  $\frac{\Gamma}{\varphi}$ называется \textit{корректным}, если
  $\Gamma \models \varphi$.
\end{definition}

Пример корректных правил: $\ds \frac{p}{p}$, $\ds \frac{p \im q, q
\im r}{p \im r}$ (силлогизм); пример некорректных правил: $\ds
\frac{}{p}$, $\ds \frac{p}{p \land q}$.

\begin{definition}
  Правило вывода $\ds\frac{\Delta}{\psi}$ является частным случаем
  правила $\ds\frac{\Gamma}{\phi}$, если существуют формулы $\theta_1,
  \dots, \theta_n$ и переменные $p_1, \dots, p_n$, такие что первое
  правило получается из второго путём одновременной подстановки
  формул $\theta_i$ вместо каждого вхождения переменной $p_i$ во всех
  посылках правила $\psi$ (с сохранением их порядка), а также в его заключении.
\end{definition}

Например, $\ds\frac{(x \im x) \land (\n y)}{x \im x}$ --- частный
случай правила $\ds\frac{p \land q}{p}$.

Мы будем рассматривать наши правила выводов как схемы, т.е. одно
правило --- по сути бесконечно много правил, включающее все частные
случаи данного правила.

\begin{definition}
  Пусть у нас есть множество правил вывода $\mathcal{R}$,
  \textit{выводом} в $\mathcal{R}$ из множества \textit{гипотез}
  $\Gamma$ будем называть последовательность формул, каждая из
  которых либо принадлежит $\Gamma$, либо получена с помощью частного
  сдучая некоторого правила из $\mathcal{R}$, при этом множество
  посылок должно состоять только из формул, которые появлялись в выводе раньше.

  Формула $\varphi$ \textit{выводится} из $\Gamma$ в $\mathcal{R}$
  ($\Gamma \vdash_{\mathcal{R}} \varphi$), если существует вывод из
  $\Gamma$ в $\mathcal{R}$, заканчивающийся формулой $\varphi$.

  Если формула $\varphi$ выводится из пустого множества гипотез в
  $\mathcal{R}$, то мы говорим, что $\varphi$ выводима в
  $\mathcal{R}$ и записывается как $\vdash_{\mathcal{R}} \varphi$.
\end{definition}

Пример:
\[
  \mathcal{R} = \set{\frac{p \im q, q \im r}{p \im r}, \frac{p}{\n\n
  p}},\ \Gamma = \set{p \im \n p}
\]

Тогда примером вывода будет:
\begin{align*}
  &p \im \n p\\
  &\n\n(p \im \n p)
\end{align*}

Первое правило из $\mathcal{R}$ мы использовать для вывода не можем.
Добавим в $\Gamma$ гипотезу $\n p \im q$. Тогда, можем дополнить вывод до
\begin{align*}
  &\n p \im q\\
  &p \im q.
\end{align*}

Таким образом, $\Gamma \vdash_{\mathcal{R}} p \im q$

\begin{theorem}[Теорема о корректности]
  Если все правила в $\mathcal{R}$ корректны и $\Gamma
  \vdash_{\mathcal{R}} \varphi$, то $\Gamma \models \varphi$.
  \begin{proof}
    По индукции. Знаем, что для $\varphi_1, \dots, \varphi_n \colon
    \forall i \colon \Gamma \models \varphi_i$.

    \textbf{База}: $i = 1 \implies$
    \begin{enumerate}
      \item $\varphi_1 \in \Gamma \implies \varphi_1$ истинная в модели;

      \item $\ds\frac{}{\varphi_1}$ --- частный случай правила из
        $\mathcal{R}$. Отсюда $\varphi_1$ --- тавтология $\implies
        \mathcal{M} \models \varphi_1$.
    \end{enumerate}

    \textbf{Шаг}: пусть $\forall j < i \colon \Gamma \models
    \varphi_j$. Докажем для $\varphi_i$.

    \begin{enumerate}
      \item $\varphi_i \in \Gamma \implies \mathcal{M} \models \varphi_i$

      \item $\exists j_1, \dots, j_k < i \colon \ds\frac{\varphi_{j_1},
        \dots, \varphi_{j_k}}{\varphi_i}$ --- частный случай правила
        из $\mathcal{R}$. Тогда по предположению индукции
        $\varphi_{j_1}, \dots, \varphi_{j_k}$ истинны в $\mathcal{M}$.

        \begin{lemma}
          Если $\ds\frac{\varphi_1, \dots, \varphi_n}{\varphi}$
          корректно и $\ds\frac{\eta_1, \dots, \eta_n}{\eta}$ ---
          частный случай $(P)$, то оно тоже корректно.
          \begin{proof}
            Имеем $\eta_1 = \varphi_1[p_1/\theta_1, p_2/\theta_2, \dots,
            p_n /\theta_n], \dots, \eta = \varphi[p_1/\theta_1, p_2/\theta_2,
            \dots, p_n/\theta_n]$

            $P$ --- корректна $\implies$ $\forall M \colon \varphi_1,
            \dots, \varphi_n$ истинны $\implies \varphi$ --- истинна.

            Пусть $\eta_1, \dots, \eta_n$ истинны в модели $M$.
            Возьмём $M'$ такую, что $\forall i \colon M' \models p_i
            \iff \mathcal{M} \models \theta_i$.

            Утверждение: $\forall \varphi \colon \mathcal{M} \models
            \varphi \iff \mathcal{M} \models \varphi[p / \theta_1]$.
            Доказывается индукцией по длине $\varphi$.
          \end{proof}
        \end{lemma}
        Тогда по лемме $\varphi_i$ истинна в $\mathcal{M}$.
    \end{enumerate}
  \end{proof}
\end{theorem}

\subsection{Конкретные правила}

\begin{definition}
  \textit{Modus Ponens}:
  \[
    \frac{p, p \im q}{q}
  \]
\end{definition}

\begin{definition}
  \textit{Аксиомы Гильберта}:
  \begin{enumerate}
    \item $(I1)$:
      \[
        q \im (p \im q)
      \]

    \item $(D)$:
      \[
        (p \im (q \im r)) \im ((p \im q) \im (p \im r))
      \]

    \item $(N)$:
      \[
        (\n q \im \n p) \im (p \im q)
      \]
  \end{enumerate}
\end{definition}

\pagebreak

\section{Лекция 6 (продолжение про выводы)}

\begin{definition}
  Правило $\frac{\Gamma}{\varphi}$ называется \textit{допустимым} в
  множестве правил вывода $\mathcal{R}$, если
  \begin{displaymath}
    \Gamma \vdash_{\mathcal{R}} \varphi
  \end{displaymath}
\end{definition}

\begin{lemma}
  Если $\frac{\Gamma}{\varphi}$ --- допустимое в $\mathcal{R}$
  правило, а $\frac{\Delta}{\psi}$ --- частный случай правила
  $\frac{\Gamma}{\varphi}$, то
  \begin{displaymath}
    \Delta \vdash_{\mathcal{R}} \psi
  \end{displaymath}
  \begin{proof}
    Пусть есть вывод $\Gamma \vdash_{\mathcal{R}} \varphi$:
    \begin{enumerate}
      \item $\xi_1$

      \item $\dots$

      \item $\xi_n = \varphi$
    \end{enumerate}
    Существует подстановка, переводящая $\Gamma$ в $\Delta$ и
    $\varphi$ в $\psi$. Сделаем такую подстановку во все формулы вывода.
  \end{proof}
\end{lemma}

\begin{theorem}[The Lemma Theorem]
  Если правило $\rho$ допустимо (доказуемо) в множестве правил вывода
  $\mathcal{R} \un \lambda$ и при этом $\lambda$ допустимо в
  $\mathcal{R}$, то и $\rho$ допустимо в $\mathcal{R}$.
  \begin{proof}
    Пусть $\ds\lambda = \frac{\Delta}{\psi}$, $\ds\rho =
    \frac{\Gamma}{\varphi}$, а также два вывода:
    \begin{itemize}
      \item $\Gamma \vdash_{\mathcal{R} \un \set{\lambda}} \varphi$:
        \begin{enumerate}
          \item $\xi_1$

          \item $\dots$

          \item $\xi_n = \varphi$
        \end{enumerate}

      \item $\Delta \vdash_{\mathcal{R}} \psi$:
        \begin{enumerate}
          \item $\eta_1$

          \item $\dots$

          \item $\eta_m = \psi$
        \end{enumerate}
    \end{itemize}
    Если в выводе $\xi_i$ получено по правилу из $\mathcal{R}$, то
    всё хорошо. Если же $\xi_i$ получено с помощью $\lambda$, то
    имеем правило вывода $\ds\frac{\xi_{j_1}, \dots,
    \xi_{j_l}}{\xi_i}$ для $\xi_{j_1}, \dots, \xi_{j_l} \in
    \set{\xi_1, \dots, \xi_{i-1}}$, частный случай $\lambda$. Тогда
    по предыдущей лемме можем этот вывод заменить на вывод
    $\xi_{j_1}, \dots, \xi_{j_l} \vdash_{\mathcal{R}} \xi_i$:
    \begin{enumerate}
      \item $\eta'_1$

      \item $\dots$

      \item $\eta'_k = \xi_i$
    \end{enumerate}
    и вставить его в доказательство вместо $\xi_i$. Делаем такую
    подстановку во все такие строки вывода и получаем новый вывод
    $\Gamma \vdash_{\mathcal{R}} \varphi$.
  \end{proof}
\end{theorem}

\subsection{Примеры допустимых правил}
Пусть $\mathcal{R} = \set{I1, D, N, MP}$, тогда в $\mathcal{R}$
допустимы правила
\begin{displaymath}
  (I0)\frac{}{p \im p},\ \frac{p \im q, q \im r}{p \im r},\ \frac{p
  \im q, p \im \n q}{\n p}
\end{displaymath}

\begin{theorem}[Теорема о дедукции]
  Если $\mathcal{R}$ --- множество правил вывода, содержащее $MP$,
  $I1$ и $D$, и все остальные правила являются аксиомами (пустое
  множество посылок), то для любых формул $\varphi$, $\psi$ и
  множества формул $\Gamma$ верно
  \begin{displaymath}
    \Gamma \un \set{\varphi} \vdash_{\mathcal{R}} \psi \iff \Gamma
    \vdash_{\mathcal{R}} (\varphi \im \psi)
  \end{displaymath}
  \begin{proof}
    В одну сторону можно усилить утверждение

    \begin{lemma}
      Если $\mathcal{R}$ --- множество правил вывода, содержащее $MP$,
      то для любых формул $\varphi$, $\psi$ и множества формул $\Gamma$ верно
      \[
        \Gamma \vdash_{\mathcal{R}} (\varphi \im \psi) \implies \Gamma
        \un \set{\varphi} \vdash_{\mathcal{R}} \psi
      \]
      \begin{proof}
        Имеем изначальный вывод $\Gamma \vdash_{\mathcal{R}} (\varphi \im \psi)$
        \begin{enumerate}
          \item $\dots$

          \item $\varphi \im \psi$
        \end{enumerate}
        Тогда получаем следующий вывод $\Gamma \un \set{\varphi}
        \vdash_{\mathcal{R}} \psi$:
        \begin{enumerate}
          \item $\dots$

          \item $\varphi \im \psi$

          \item $\psi$ (MP)
        \end{enumerate}
      \end{proof}
    \end{lemma}

    Осталось доказать, что $\Gamma \un \set{\varphi}
    \vdash_{\mathcal{R}} \psi \implies \Gamma \vdash_{\mathcal{R}}
    (\varphi \im \psi)$.

    Будем доказывать индукцией по длине вывода. Пусть имеем вывод
    $\Gamma \un \set{\varphi} \vdash_{\mathcal{R}} \psi$:
    \begin{enumerate}
      \item $\xi_1$

      \item $\dots$

      \item $\xi_n = \psi$
    \end{enumerate}
    Будем доказывать, что для любого $i \leqslant n$ верно
    \[
      \Gamma \vdash_{\mathcal{R}} (\varphi \im \xi_i)
    \]
    Разберём случаи:
    \begin{itemize}
      \item Возьмём $\xi_i = \varphi$. Докажем, что $\Gamma
        \vdash_{\mathcal{R}} (\varphi \im \varphi)$. Но уже
        доказывали (в ДЗ), что из пустого множества посылок доказуемо
        выражение $p \im p$, а $\varphi \im \varphi$ --- его частный случай.

      \item Пусть теперь $\xi_i \in \Gamma$. Тогда нужно построить
        вывод $\Gamma \vdash_{\mathcal{R}} (\varphi \im \xi_i)$. Тогда
        \begin{enumerate}
          \item $\xi_i$

          \item $\xi_i \im (\varphi \im \xi_i)$ (I1)

          \item $\varphi \im \xi_i$ (MP)
        \end{enumerate}

      \item $\xi_i$ получена по правилу I1 или D (с пустым
        множеством посылок). Вывод будет тот же самый, но первой
        строкой вывода $\xi_i$ записан не потому что $\xi_i \in
        \Gamma$, а потому что $\xi_i$ выводится из аксиом I1 и D.

      \item $\xi_i$ получена по правилу MP из $\xi_j$ и $\xi_k$.
        Тогда $\xi_k$ имеет вид $\xi_j \im \xi_i$. По предположению
        индукции $\Gamma \vdash_{\mathcal{R}} (\varphi \im \xi_j)$ и
        $\Gamma \vdash_{\mathcal{R}} \underbrace{(\varphi \im (\xi_j
        \im \xi_i))}_{\xi_k}$. Тогда получаем следующий вывод:
        \begin{enumerate}
          \item $\dots$

          \item $\varphi \im \xi_j$

          \item $\dots$

          \item $\varphi \im (\xi_j \im \xi_i)$

          \item $(\varphi \im (\xi_j \im \xi_i)) \im ((\varphi \im
            \xi_j) \im (\varphi \im \xi_i))$ (D)

          \item $((\varphi \im \xi_j) \im (\varphi \im \xi_i))$ (MP)

          \item $(\varphi \im \xi_i)$ (MP)
        \end{enumerate}
    \end{itemize}
  \end{proof}
\end{theorem}

\subsection{Противоречивость и непротиворечивость}

Рассмотрим аксиому
\begin{displaymath}
  (I2)\ (\n p \im (p \im q))
\end{displaymath}

Пусть $\mathcal{R}$ --- множество правил вывода, которое включает MP,
I0 и I2, а также может дополнительно включать только правила вывода
без посылок. Множество формул $\Gamma$ называется (синтаксически)
\textit{противоречивым} (\textit{inconsistent}) (относительно
$\mathcal{R}$), если
выполняется одно из следующих пяти эквивалентных условий:
\begin{itemize}
  \item Формула $\n(p \im p)$ выводится из $\Gamma$ в $\mathcal{R}$
    ($\Gamma \vdash_{\mathcal{R}} \n (p \im p))$

  \item Отрицание некоторой аксиомы выводимо из $\Gamma$

  \item $\Gamma \vdash_{\mathcal{R}} \varphi$ и $\Gamma
    \vdash_{\mathcal{R}} \n \varphi$ для некоторой формулы $\varphi$

  \item Из $\Gamma$ можно вывести в $\mathcal{R}$ любую формулу (вообще любую)

  \item Отрицание всех аксиом доказуемы в $\mathcal{R}$ из $\Gamma$
\end{itemize}

Множество формул, которое не является противоречивым, называется
\textit{непротиворечивым} (\textit{consistent}).

\begin{proof}
  $1 \implies 2$: $\n (p \im p)$ само по себе есть отрицание аксиомы I0.

  $2 \implies 3$:
  Пусть есть $\alpha$ --- аксиомы из $\mathcal{R}$ и $\varphi =
  \alpha$, тогда $\Gamma \vdash_{\mathcal{R}} \alpha$. С другой
  стороны, $\Gamma \vdash_{\mathcal{R}} \n \alpha$ из предположения.

  $3 \implies 4$:
  Имеем вывод $\Gamma \vdash_{\mathcal{R}} \varphi$ и $\Gamma
  \vdash_{\mathcal{R}} \n \varphi$. Объединим их в один вывод:
  \begin{enumerate}
    \item $\dots$

    \item $\varphi$

    \item $\dots$

    \item $\n \varphi$
  \end{enumerate}

  Тогда дополним его:
  \begin{enumerate}
      \setcounter{enumi}{4}

    \item $\n \varphi \im (\varphi \im \psi)$ (I2)

    \item $\varphi \im \psi$ (MP)

    \item $\psi$ (MP)
  \end{enumerate}

  $4 \implies 5$: можно вывести любую формулу, значит можно вывести и
  отрицания аксиом.

  $5 \implies 1$: отрицания всех аксиом доказуемы, значит доказуемо и
  отрицание I0.
\end{proof}

\begin{theorem}
  Пусть $\set{MP, I1, D, N} \sus \mathcal{R}$ и, кроме MP, все
  правила в $\mathcal{R}$ без посылок. Для любого множества формул
  $\Gamma$ и формулы $\varphi$ верно, что $\Gamma \un \set{\n
  \varphi}$ противоречиво в $\mathcal{R}$, тогда $\Gamma
  \vdash_{\mathcal{R}} \varphi$. Т.е.,
  \begin{displaymath}
    \Gamma \un \set{\n \varphi} \vdash_{\mathcal{R}} \n (p \im p)
    \implies \Gamma \vdash_{\mathcal{R}} \varphi
  \end{displaymath}
  \begin{proof}
    По теореме о дедукции, левая часть последнего утверждения
    равносильна $\Gamma \vdash_{\mathcal{R}} (\n \varphi \im \n (p \im p))$.

    Тогда имеем вывод
    \begin{enumerate}
      \item $\dots$

      \item $\n \varphi \im \n (p \im p)$

      \item $(\n \varphi \im \n (p \im p)) \im ((p \im p) \im \varphi)$ (N)

      \item $(p \im p) \im \varphi$ (MP)

      \item $\dots$ (вывод I0 с помощью $\set{\mathrm{MP},
        \mathrm{I1}, \mathrm{D}, \mathrm{N}}$)

      \item $p \im p$ (I0)

      \item $\varphi$
    \end{enumerate}
  \end{proof}
\end{theorem}

\begin{lemma}
  Пусть $\mathcal{R} = \set{MP, I1, D, N}$, тогда правило I2
  допустимо в $\mathcal{R}$, т.е.
  \begin{displaymath}
    \vdash_{\mathcal{R}} (\n p \im (p \im q))
  \end{displaymath}
  \begin{proof}
    По теореме о дедукции вместо изначального утверждения можем
    доказать $\n p \vdash_{\mathcal{R}} (p \im q)$. Построим вывод.
    Гипотеза --- $\n p$, тогда
    \begin{enumerate}
      \item $\n p$

      \item $\n p \im (\n q \im \n p)$ (I1)

      \item $\n q \im \n p$ (MP)

      \item $(\n q \im \n p) \im (p \im q)$ (N)

      \item $p \im q$ (MP)
    \end{enumerate}
  \end{proof}
\end{lemma}

\pagebreak

\section{Лекция 7}

\subsection{Основное множество правил}

\begin{align*}
  (MP)\ &\frac{p, p \im q}{q}\\
  (I0)\ &(p \im p)\\
  (I1)\ &(q \im (p \im q))\\
  (D)\ &((p \im (q \im r)) \im ((p \im q) \im (p \im r)))\\
  (I2)\ &(\n p \im (p \im q))\\
  (N)\ &((\n q \im \n p) \im (p \im q))\\
  (NI)\ &(p \im (\n q \im \n (p \im q)))\\
  (NN)\ &(p \im \n\n p)\\
  (R)\ &((q \im p) \im ((\n q \im p) \im p))
\end{align*}
Будем писать $\vdash$ (без индекса) для выводимости в этом множестве правил.

На самом деле, обязательными являются лишь правила I1, D, N.

Обозначим систему аксиом Гильберта как $\mathcal{H}$.

\begin{definition}
  Пусть $\varphi$ --- формула, и $b \in \set{\mathrm{True},
  \mathrm{False}}$, тогда
  \[
    \varphi^b =
    \begin{cases}
      \varphi,\ b = \mathrm{True}\\
      \n \varphi,\ b = \mathrm{False}
    \end{cases}
  \]
\end{definition}

\begin{definition}
  Пусть $M$ --- некоторая конечная модель , т.е. отображение из
  конечного множества переменных в множество $\set{\mathrm{True},
  \mathrm{False}}$.

  Определим множество формул
  \[
    \Gamma_M = \bigcup_{M[p] = b} \set{p^b}
  \]
\end{definition}

Например, если $M = \set{p\colon \mathrm{True}, q\colon
\mathrm{False}, x\colon \mathrm{False}}$, то $\Gamma_M = \set{p, \n q, \n x}$.

\begin{lemma}
  Пусть $\varphi$ --- формула, и $M$ оценивает все формулы из
  $\varphi$ и $M[\varphi]$ --- истинностное значение формулы
  $\varphi$ при оценке $M$, тогда
  \[
    \Gamma_M \vdash \varphi^{M[\varphi]}
  \]
  \begin{proof}
    Доказательство индукцией по длине формулы:

    \textbf{База}: $\varphi = p$. Утверждение следует из того, что $p
    \vdash p$ и $\n p \vdash \n p$.

    \textbf{Шаг}:
    \begin{itemize}
      \item $\varphi = \n \psi$

        Если $M[\psi] = \mathrm{True}$, то $M[\varphi] -
        \mathrm{False}$. Тогда по предположению индукции $\Gamma_M
        \vdash \psi^{\mathrm{True}} = \psi$. Надо доказать, что
        $\Gamma_M \vdash \varphi^{\mathrm{False}} = \n \n \psi$.
        Достаточно доказать, что $\vdash (\psi \im \n \n \psi)$
        (правило NN). Вывод для $\n \n \psi$ выглядит так:
        \begin{enumerate}
          \item $\dots$

          \item $\psi$

          \item $\psi \im \n \n \psi$ (NN)

          \item $\n \n \psi$
        \end{enumerate}

        Если $M[\psi] = \mathrm{False}$, то $M[\varphi] =
        \mathrm{True}$. По ПИ имеем $\Gamma_M \vdash \n \psi$, нужно
        доказать $\Gamma_M \vdash \varphi = \n \psi$. Вывод будет тривиальным;
        \begin{enumerate}
          \item $\n \psi$

          \item $\varphi$
        \end{enumerate}

      \item $\varphi = (\psi_1 \im \psi_2)$. По предположению
        индукции $\Gamma_M \vdash \psi_1^{M[\psi_1]}$ и $\Gamma_M
        \vdash \psi_2^{M[\psi_2]}$. Надо разобрать 4 случая:
        \begin{enumerate}
          \item $M[\psi_1] = \mathrm{False}, M[\psi_2] =
            \mathrm{False}$. Докажем $\n \psi_1, \n \psi_2 \vdash
            (\psi_1 \im \psi_2)$ с помощью I2 ($\n p \im (p \im q)$).
            Имеем вывод
            \begin{enumerate}
              \item \dots

              \item $\n \psi_1$

              \item $\n \psi_1 \im (\psi_1 \im \psi_2)$ (I2)

              \item $\psi_1 \im \psi_2$ (MP)
            \end{enumerate}

          \item False, True. Докажем $\n \psi_1, \psi_2 \vdash
            (\psi_1 \im \psi_2)$ (I2) или (I1). То же самое, т.к.
            $\psi_2$ в прошлом выводе не участвовал.

          \item True, False. Докажем $\psi_1, \n \psi_2 \vdash \n
            (\psi_1 \im \psi_2)$ (NI). Получим вывод
            \begin{enumerate}
              \item \dots

              \item $\psi_1$

              \item \dots

              \item $\n \psi_2$

              \item $\psi_1 \im (\n \psi_2 \im \n(\psi_1 \im \psi_2))$ (NI)

              \item $\n \psi_2 \im \n (\psi_1 \im \psi_2)$ (MP)

              \item $\n (\psi_1 \im \psi_2)$ (MP)
            \end{enumerate}

          \item True, True. Докажем $\psi_1, \psi_2 \vdash (\psi_1
            \im \psi_2)$. Соответствующий вывод:
            \begin{enumerate}
              \item \dots

              \item $\psi_2$

              \item $\psi_2 \im (\psi_1 \im \psi_2)$ (I1)

              \item $\psi_1 \im \psi_2$ (MP)
            \end{enumerate}
        \end{enumerate}
    \end{itemize}
  \end{proof}
\end{lemma}

\begin{lemma}
  Если $\Gamma \un \set{p} \vdash \varphi$ и $\Gamma \un \set{\n p}
  \vdash \varphi$, то $\Gamma \vdash \varphi$.
  \begin{proof}
    Следует из аксиомы (R).

    По теореме о дедукции из первого утверждения $\Gamma \vdash p \im
    \varphi$, а из второго $\Gamma \vdash \n p \im \varphi$. Получим вывод:
    \begin{enumerate}
      \item $\dots$

      \item $p \im \varphi$

      \item $\dots$

      \item $\n p \im \varphi$

      \item $(p \im \varphi) \im ((\n p \im \varphi) \im \varphi)$ (R)

      \item $(\n p \im \varphi) \im \varphi$ (MP)

      \item $\varphi$
    \end{enumerate}
  \end{proof}
\end{lemma}

\begin{theorem}[О полноте в слабой форме]
  Если $\varphi$ --- тавтология, то $\vdash \varphi$, а значит и
  $\vdash_{\mathcal{H}} \varphi$.
  \begin{proof}
    $\forall$ модели $M$, содержащей все переменные из $\varphi$
    имеем $M[\varphi] = \mathrm{True}$, Тогда по лемме
    \[
      \Gamma_M \vdash \varphi\ \text{для любой модели $M$.}
    \]
    Пусть $p$ --- некоторая переменная из $\varphi$, тогда все модели
    разобьются на пары, т.что в паре оценка отличается только в
    переменной $p$. Пусть $M_1$ и $M_2$ --- две такие модели. Пусть
    \[
      \Gamma_{M_1} = \Gamma' \un \set{p}\ \text{и}\ \Gamma_{M_2} =
      \Gamma' \un \set{\n p}
    \]
    По предыдущей лемме получим $\Gamma' \vdash \varphi$. Проделав
    так с каждой парой моделей мы уменьшим на 1 количество посылок.
    Действуя так мы сможем избавиться от всех посылок.
  \end{proof}
\end{theorem}

\begin{theorem}[О полноте в сильной форме]
  Пусть $\Gamma$ --- конечное множество формул и $\varphi$ --- формула, тогда
  \[
    \Gamma \models \varphi \iff \Gamma \vdash \varphi
  \]
  \begin{proof}
    $\impliedby$ было в теореме о корректности.

    $\implies$ Пусть $\Gamma = \set{\psi_1, \dots, \psi_n}$. По
    теореме о дедукции (применив $n$ раз)
    \[
      \Gamma \vdash \varphi \iff \vdash (\psi_1 \im (\dots (\psi_n
      \im \varphi)\dots))
    \]
    Осталось показать, что
    \[
      \Gamma \models \varphi \iff (\psi_1 \im (\dots(\psi_n \im
      \varphi)\dots))\ \text{--- тавтология}
    \]
    В правую сторону импликация известна, осталось доказать в левую.
    Пусть это не тавтология, тогда $\exists$ модель, её опровергащая.
    Тогда надо чтобы $\psi_i$ были истинными, а $\varphi$ --- ложной.
    Но тогда $\Gamma \not\models \varphi$.
  \end{proof}

  Переформулируем это утверждение в симметричной форме.

  $\Gamma \models \varphi$ эквивалентно тому, что $\Gamma \un \set{\n
  \varphi}$ не имеет модели.

  $\Gamma \vdash \varphi$ эквивалентно тому, что $\Gamma \un
  \set{\n \varphi}$ противоречиво:
  \[
    \Gamma \un \set{\n \varphi} \vdash \n \varphi, \Gamma \un \set{\n
    \varphi} \vdash \varphi \implies \Gamma \un \set{\n
    \varphi}\ \text{--- противоречиво}
  \]
  В другую сторону знаем, что $\Gamma \un \set{\n \varphi} \vdash \n
  (p \im p)$. Тогда:
  \begin{enumerate}
    \item $\Gamma \vdash \n \varphi \im \n (p \im p)$

    \item $(\n \varphi \im \n (p \im p)) \im ((p \im p) \im \varphi)$ (N)

    \item $(p \im p) \im \varphi$

    \item $p \im p$ (I0)

    \item $\varphi$
  \end{enumerate}

  Тогда изначальное утверждение эквивалентно следующему:
  \begin{center}
    $\Gamma$ не имеет модели $\iff$ $\Gamma$ противоречиво
  \end{center}
  или
  \begin{center}
    $\Gamma$ выполнимо $\iff$ $\Gamma$ непротиворечиво
  \end{center}
\end{theorem}

\begin{theorem}[О компактности (синтаксическая)]
  Бесконечное множество формул непротиворечиво тогда и только тогда,
  когда любое его конечное подмножество непротиворечиво.
  \begin{proof}
    Пусть $\Gamma$ противоречиво, тогда можно вывести $\Gamma \vdash
    \n (p \im p)$, т.е. имеем вывод
    \begin{enumerate}
      \item \dots

      \item $\n (p \im p)$,
    \end{enumerate}
    он использует конечное число формул из $\Gamma$. Пусть $\Gamma_0$
    --- все формулы, используемые в доказательстве. Тогда $\Gamma_0
    \vdash \n (p \im p)$.

    В другую сторону, если существует $|\Gamma_0| < \infty, \Gamma_0
    \subseteq \Gamma$ и $\Gamma_0$ противоречива, то $\Gamma$ также
    противоречива.
  \end{proof}
\end{theorem}

\begin{theorem}[О компактности (семантическая)]
  Бесконечное подмножество формул $\Gamma$ выполнимо (имеет модель)
  тогда и только тогда, когда любое его конечное подмножество выполнимо.
\end{theorem}
\pagebreak

\section{Лекция 8}

\begin{theorem}[О полноте в сильной форме]
  Произвольное множество формул $\Gamma$ выполнимо тогда и только
  тогда, когда $\Gamma$ непротиворечиво.

  \begin{proof}
    \begin{definition}
      Множество формул $\Gamma$ называется \textit{полным}, если оно
      непротиворечиво и <<максимально>>, т.е. для любой формулы $\varphi
      \nin \Gamma$ верно, что $\Gamma \un \set{\varphi}$ --- противоречиво.
    \end{definition}

    \begin{lemma}[Линденбаум]
      Любое непротиворечивое множество можно дополнить до полного.
      \begin{proof}
        Перечислим все формулы $\varphi_0, \varphi_1, \dots$

        Будем строить $\Gamma_n$ по индукции. $\Gamma_0 = \Gamma$,
        \[
          \Gamma_{n+1} =
          \begin{cases}
            \Gamma_n \un \set{\varphi_n},\ \text{если $\Gamma_n \un
            \set{\varphi_n}$ --- непротиворечиво}\\
            \Gamma_n,\ \text{иначе}
          \end{cases}
        \]
        \[
          \Delta = \bigcup_{n \in \mathbb{N}} \Gamma_n
        \]
        Проверим, что $\Delta$ непротиворечиво и максимально.
        \begin{itemize}
          \item Непротиворечивость

            Если $\Delta$ противоречиво, то $\Delta \vdash \n (p \im p)$.
            Такой вывод использует конечное число формул, а значит
            $\exists n \colon \Gamma_n \vdash \n (p \im p) \implies
            \Gamma_n$ --- противоречиво, что противоречит построению $\Gamma_n$

          \item Максимальность

            Пусть оно не максимально, т.е. $\exists \varphi \nin \Delta
            \colon \Delta \un \set{\varphi}$ --- непротиворечиво. Тогда
            $\exists n \colon \varphi = \varphi_n$, но тогда
            $\Gamma_{n+1} = \Gamma_n \un \set{\varphi_n}$
            \begin{enumerate}
              \item $\Gamma_n \un \set{\varphi_n}$ --- противоречиво,
                тогда $\Gamma_n \un \set{\varphi_n} \vdash \n (p \im p)$,
                откуда следует $\Delta \vdash \n (p \im p)$, а значит
                $\Delta$ противоречива, противоречие

              \item $\Gamma_n \un \set{\varphi_n}$ --- непротиворечиво,
                тогда $\varphi_n = \varphi \in \Delta$, противоречие
            \end{enumerate}
        \end{itemize}
      \end{proof}
    \end{lemma}

    \begin{lemma}[Свойства полных множеств]
      Пусть $\Delta$ --- полное множество формул, тогда:
      \begin{itemize}
        \item $\n \psi \in \Delta \iff \psi \nin \Delta$

        \item $(\psi_1 \im \psi_2) \in \Delta \iff (\psi_1 \nin \Delta
          \lv \psi_2 \in \Delta)$
      \end{itemize}
      \begin{proof}
        \begin{itemize}
          \item Пусть оба $\in \Delta$, тогда получаем $\Delta \vdash
            \psi$ и $\Delta \vdash \n \psi$ и $\Delta$ противоречиво

            Пусть оба $\nin \Delta$. Тогда по максимальности добавление
            $\psi$ и $\n \psi$ приводит к противоречивости:
            \[
              \begin{cases}
                \Delta \un \set{\psi}\ \text{--- противоречиво} \implies
                \Delta \vdash \n \psi\\
                \Delta \un \set{\n \psi}\ \text{--- противоречиво}
                \implies \Delta \vdash \n\n \psi\\
              \end{cases}
            \]

          \item Имея первый пункт, достаточно исключить следующие 3 случая:
            \begin{itemize}
              \item $(\psi_1 \im \psi_2) \in \Delta$ и $\psi_1 \in
                \Delta$ и $\n \psi_2 \in \Delta$, тогда $\Delta \vdash
                (\psi_1 \im \psi_2)$ и $\Delta \vdash \psi_1$. По MP
                получаем $\Delta \vdash \psi_2$, но $\Delta \vdash \n
                \psi_2$, откуда $\Delta$ противоречиво.

              \item $\n(\psi_1 \im \psi_2) \in \Delta$ и $\n \psi_1 \in
                \Delta$ (равносильно $\psi_1 \nin \Delta$ по первому
                пункту). Имеем $\n \psi_1 \im (\psi_1 \im \psi_2)$ по I2;
                по MP получим $\Delta \vdash \psi_1 \im \psi_2$, и
                имея $\Delta \vdash \n (\psi_1 \im \psi_2)$ получаем
                противоречие.

              \item $\n(\psi_1 \im \psi_2) \in \Delta$ и $\psi_2 \in
                \Delta$. Вывод аналогичен предыдущему пункту:
                \begin{enumerate}
                  \item $\psi_2 \im (\psi_1 \im \psi_2)$ (I1)

                  \item $\psi_1 \im \psi_2$ (MP)

                  \item $\n (\psi_1 \im \psi_2)$ по условию, противоречие.
                \end{enumerate}
            \end{itemize}
        \end{itemize}
      \end{proof}
    \end{lemma}

    \begin{definition}
      Пусть $\Delta$ --- полное множество формул. Определим модель
      $M_\Delta$ следующим образом:
      \[
        M_\Delta[p] =
        \begin{cases}
          \mathrm{True},\ p \in \Delta\\
          \mathrm{False},\ p \nin \Delta\\
        \end{cases}
      \]
    \end{definition}

    \begin{lemma}
      Для произвольной формулы $\varphi$ верно, что
      \[
        M_\Delta[\varphi] = \mathrm{True} \iff \varphi \in \Delta
      \]
      \begin{proof}
        Докажем индукцией по $\varphi$.

        \textbf{База}: верна по определению $M_\Delta[\varphi]$

        \textbf{Шаг}:
        \begin{itemize}
          \item $\varphi = \n \psi$

            $M_\Delta[\varphi] =
            \begin{cases}
              \mathrm{True},\ M_\Delta[\psi] = \mathrm{False} \iff \psi
              \nin \Delta \iff \varphi = \n \psi \in \Delta\\
              \mathrm{False},\ M_\Delta[\psi] = \mathrm{True} \iff \psi
              \in \Delta \iff \varphi = \n \psi \nin \Delta
            \end{cases}$ из предыдущей леммы.

          \item $\varphi = (\psi_1 \im \psi_2)$.

            $(\psi_1 \im \psi_2) \in \Delta \iff (\psi_1 \nin \Delta \lor
            \psi_2 \in \Delta) \iff M_\Delta[\psi_1] = \mathrm{False} \lor
            M_\Delta[\psi_2] = \mathrm{True} \iff M_\Delta[\psi_1 \im
            \psi_2] = \mathrm{True}$
        \end{itemize}
      \end{proof}
    \end{lemma}

    Докажем наконец изначальную теорему (да, это было доказательство
    теоремы на полторы страницы):

    $\implies$ Если $\Gamma$ выполнимо, то $\exists M$, оценивающая все
    переменные, в которой все формулы из $\Gamma$ истинны.

    Пусть $\Gamma$ противоречиво, тогда $\Gamma \vdash \n (p \im p)$,
    но в силу того, что все правила корректны, т.е. сохраняют
    истинность, то в модели $M$ должна быть истинная формула $\n(p \im
    p)$, что невозможно.

    $\impliedby$ Теперь, пусть $\Gamma$ непротиворечиво, тогда его
    можно расширить по лемме Линденбаума до полного множества $\Delta$.
    Тогда в модели $M_\Delta$ будут истинны все формулы из $\Gamma$
    благодаря предыдущей лемме.
  \end{proof}
\end{theorem}

\subsection{Доказательства аксиом}

Напоминание про силлогизм:
\[
  \frac{p \im q, q \im r}{p \im r}
\]

\begin{itemize}
  \item (I0):
    \[
      p \im p
    \]
    \begin{proof}
      \begin{enumerate}
        \item $p \im ((p \im p) \im p)$ (I1)

        \item $(p \im ((p \im p) \im p)) \im ((p \im (p \im p)) \im
          (p \im p))$ (D)

        \item $(p \im (p \im p)) \im (p \im p)$ (MP)

        \item $p \im (p \im p)$ (I1)

        \item $p \im p$ (MP)
      \end{enumerate}
    \end{proof}
  \item (I2):
    \[
      \n p \im (p \im q)
    \]
    \begin{proof}
      \begin{enumerate}
        \item $\n p \im (\n q \im \n p)$ (I1)

        \item $(\n q \im \n p) \im (p \im q)$ (N)

        \item $\n p \im(p \im q)$ (силлогизм)
      \end{enumerate}
    \end{proof}

  \item Вспомогательная аксиома
    \[
      \vdash \n\n p \im p
    \]
    \begin{proof}
      По лемме о дедукции доказательство аксиомы эквивалентно
      доказательству $\n\n p \vdash p$. Хотим вывести $\vdash \n\n p
      \im (\n\n p \im p)$
      \begin{enumerate}
        \item $\n\n p \im (\n\n\n\n p \im \n\n p)$ (I1)

        \item $\n\n p \vdash \n\n\n\n p \im \n\n p$

        \item $(\n\n\n\n p \im \n\n p) \im (\n p \im \n\n\n p)$ (N)

        \item $\n\n p \vdash \n p \im \n\n\n p$ (т.к. умеем выводить
          $\n\n\n\n p \im \n\n p$ из $\n\n p$)

        \item $(\n p \im \n\n\n p) \im (\n\n p \im p)$ (N)

        \item $\n\n p \vdash \n\n p \im p$ (т.к. умеем выводить $\n p
          \im \n\n\n p$ из $\n\n p$)

        \item $\n\n p \vdash p$
      \end{enumerate}
    \end{proof}

  \item (NN):
    \[
      p \im \n\n p
    \]
    \begin{proof}
      \begin{enumerate}
        \item $(\n\n\n p \im \n p) \im (p \im \n\n p)$ (N)

        \item $\vdash \n\n\n p \im \n p$ (предыдущий пункт, частный
          случай для $\n p$)

        \item $\vdash p \im \n\n p$ (MP)
      \end{enumerate}
    \end{proof}

  \item
    \[
      \vdash (p \im q) \im (\n q \im \n p)
    \]
    \begin{proof}
      По теореме о дедукции она выводима тогда и только тогда, когда
      выводима $(p \im q) \vdash \n q \im \n p$. Докажем вспомогательную лемму
      \begin{lemma}
        \[
          \vdash (p \im q) \im (\n\n p \im \n\n q)
        \]
        \begin{proof}
          Вывод эквивалентен $p \im q \vdash \n\n p \im \n\n q \iff p
          \im q, \n\n p \vdash \n\n q$.

          По одному из прошлых пунктов знаем $\vdash \n\n p \im p$,
          по силлогизму получаем $\vdash \n\n p \im q$, $\vdash q \im
          \n\n q$, $\vdash \n\n p \im \n\n q$.
        \end{proof}
      \end{lemma}
      Тогда получаем следующий вывод:
      \begin{enumerate}
        \item $p \im q \vdash \n\n p \im \n\n q$

        \item $p \im q \vdash \n q \im \n p$ (N)
      \end{enumerate}
    \end{proof}

  \item (NI):
    \[
      \vdash (p \im (\n q \im \n(p \im q)))
    \]
    \begin{proof}
      Достаточно вывести $p \vdash \n q \im \n (p \im q)$. Если
      сможем доказать $p \vdash (p \im q) \im q$, то получим
      требуемое (по предыдущему пункту $\n q \im \n(p \im q) \eq (p
      \im q) \im p$). По теореме о дедукции это эквивалентно $p, p
      \im q \vdash q$, по MP это верно.
    \end{proof}
\end{itemize}

\pagebreak

\section{Лекция 9 (логика предикатов)}

<<Все люди смертны, люди существуют, следовательно смертные существуют.>>:

\[
  \forall x [\mathrm{Man}(x) \im \mathrm{Mortal}(x)] \land \exists x
  [\mathrm{Man}(x)] \im \exists x [\mathrm{Mortal}(x)]
\]

Великая теорема Ферма:

\[
  \forall x [\forall y [\forall z [\forall n [n \geqslant 3 \im \n
  (x^n + y^n = z^n)]]]]
\]

\begin{definition}
  Следующие строки являются (\textit{корректными} (\textit{valid}))
  \textit{термами} в логике предикатов:
  \begin{itemize}
    \item \textbf{Имя переменной}: последовательность
      буквенно-цифровых символов, начинающаяся с буквы из диапазона
      $u-z$. Например, $x$, $y12$, $zLast$.

    \item \textbf{Имя константы}: последовательность
      буквенно-цифровых символов, начинающаяся с цифры или буквы из
      диапазона $a-e$, либо одиночный символ подчёркивания (без
      других символов до или после). Например, $0$, $e1$, $7x$, \_.

    \item \textbf{$n$-арный вызов функции} вида $f(t_1, \dots, t_n)$,
      где $f$ --- имя функции, обозначаемое последовательностью
      буквенно-цифровых символов, начинающейся с буквы из диапазона
      $f-t$, где $n \geqslant 1$, и каждый $t_i$ сам является
      (корректным) термом.
  \end{itemize}

  Примеры корректных термов: $plus(x, y)$, $s(s(0))$, $f(g(x), h(7, y), c)$.
\end{definition}

\begin{definition}
  \textit{Предикат} $P$ на множестве $\Omega \neq \varnothing$ ---
  подмножество $P \sus \Omega^n$. Например, $(A, \leqslant)$, где
  $\leqslant \sus A \x A$.
\end{definition}

\begin{definition}[Индуктивное определение формулы]
  Следующие строки являются \textit{(корректными) формулами} в логике
  предикатов:
  \begin{itemize}
    \item \textbf{Равенство} вида $t_1 = t_2$, где $t_1$, $t_2$ ---
      (корректные) термы. Например, $0 = 0$, $s(0) = 1$, $plus(x, y)
      = plus(y, x)$.

    \item \textbf{$n$-арный предикат} вида $R(t_1, \dots, t_n)$, где
      $R$ --- имя предиката, обозначаемое строкой буквенно-цифровых
      символов, начинающийся с буквы из диапазона $F-T$, где $n
      \geqslant 0$ (допускаются нульарные предикаты), и каждый $t_i$
      --- терм. Например, $R(x, y)$, $Plus(s(0), x, s(x))$, $Q()$.

    \item \textbf{Отрицание} вида $\n \phi$, где $\phi$ --- формула
      (в Python: \~{}).

    \item \textbf{Бинарная операция} вида $(\phi * \phi)$, где * ---
      один из бинарных операторов $\lor, \land, \im$, а $\phi, \psi$
      --- формулы (в Python --- |, \&, -->).

    \item \textbf{Кванторная конструкция} вида $Qx[\phi]$, где $Q$
      --- либо квантор всеобщности $\forall$ (в Python --- $A$), либо
      квантор существования $\exists$ (в Python --- $E$), $x$ --- имя
      переменной, а $\phi$ --- формула. Подформула $\phi$,
      находящаяся внутри квадратных скобок в конструкции $Qx[\phi]$,
      называется \textit{областью действия} квантора.
  \end{itemize}

  Примеры формул:
  \begin{itemize}
    \item $\forall x [x = x]$

    \item $\exists x [R(7, y)]$

    \item $\forall x [\exists y [R(x, y)]]$

    \item $\forall x [(R(x) \lor \exists x [Q(x)])]$
  \end{itemize}
\end{definition}

Есть два подхода к записи формул:
\begin{enumerate}
  \item Как у нас, есть <<полный>> набор возможных символов и
    формально все можно использовать.

  \item Мы сначала фиксируем набор символов, которые можно
    использовать (называется \textit{сигнатура}), и определяем
    формулу в данной сигнатуре.
\end{enumerate}

\begin{theorem}[Об однозначности разбора терма]
  Существует единственное дерево разбора для каждого корректного терма.
\end{theorem}

\begin{theorem}[Об однозначности разбора формулы]
  TODO
\end{theorem}

\begin{definition}[Семантика]
  \textit{Моделью} (в логике предикатов) будем называть пару $M =
  (\Omega, I)$, где $\Omega$ --- непустое множество элементов,
  которое будем называть \textit{универсумом} (\textit{носителем})
  нашей модели, а $I$ --- \textit{интерпретацией}, которая
  интерпретирует константные, функциональные и предикатные символы. При этом
  \begin{itemize}
    \item для константого символа $c$: $I(c) \in \Omega$

    \item для $n$-арного функционального символа $f$: $I(f) :
      \Omega^n \to \Omega$

    \item для $n$-арного предикатного символа $P$: $I(P) \sus \Omega^n$
  \end{itemize}
\end{definition}

\begin{definition}[Истинность формул]
  \textit{Оценка} (\textit{assignment}) $A$ --- отображение
  некоторого множества переменных в носитель модели $\Omega$.

  Значение терма $t$ в данной модели $M$ при данной оценке $A$
  определяется по индукции (это значение будем обозначать $[[t]]_{M,A}$)

  Примеры:
  \begin{itemize}
    \item $[[c]]_{M,A} = I(c)$ --- константа

    \item $[[x]]_{M,A} = A(x)$ --- переменная

    \item $[[f(t_1, \dots, t_n)]] = I(f)([[t_1]]_{M,A}, \dots,
      [[t_n]]_{M,A})$ --- функция
  \end{itemize}
\end{definition}

Как подсчитывать значение выражения:
\begin{itemize}
  \item $[[t_1 = t_2]]_{M,A} = \mathrm{True} \iff [[t_1]]_{M,A} = [[t_2]]_{M,A}$

  \item $[[P(t_1, \dots, t_n)]] = \mathrm{True} \iff ([[t_1]]_{M,A},
    \dots, [[t_n]]_{M,A}) \sus I(P)$

  \item
    \begin{itemize}
      \item $[[\forall x[\phi]]]_{M,A} = \mathrm{True} \iff
        [[\phi]]_{M,A'} = \mathrm{True}$ для всех $w \sus \Omega \colon
        A'(x) = w \land A'(y) = A(y)$ для всех переменных $y \neq x$.

      \item $[[\exists x[\phi]]]_{M,A} = \mathrm{True}$ --- то же
        самое, только вместо $\forall w$ имеем $\exists w$.
    \end{itemize}
\end{itemize}

\pagebreak

\section{Лекция 10}

\begin{definition}[Истинность формул (формулы)]
  Истинность формул $\varphi$ в данной модели $M = (\Omega, I)$ при
  данной оценке $A$ определяется по индукции:
  \begin{itemize}
    \item $[[t_1 = t_2]]_{M,A} =
      \begin{cases}
        \mathrm{True},\ \text{если $[[t_1]]_{M,A} = [[t_2]]_{M,A}$}\\
        \mathrm{False},\ \text{иначе}
      \end{cases}$

    \item $[[R(t_1, \dots, t_n)]]_{M,A} =
      \begin{cases}
        \mathrm{True},\ \text{если $([[t_1]]_{M,A}, \dots,
        [[t_n]]_{M,A}) \in [I(R)]$}\\
        \mathrm{False},\ \text{иначе}
      \end{cases}$

    \item $[[\n \varphi]]_{M,A} = \n [[\varphi]]_{M,A}$

    \item $[[\varphi * \psi]]_{M,A} = [[\varphi]]_{M,A} *
      [[\psi]]_{M,A}$, где $* \in \set{\land, \lor, \im}$

    \item $[[\forall x\varphi]]_{M,A} =
      \begin{cases}
        \mathrm{True},\ \text{если для всех $u \in \Omega \colon
        [[\varphi]]_{M,A[x \mapsto u]} = \mathrm{True}$}\\
        \mathrm{False},\ \text{иначе}
      \end{cases}$

    \item $[[\exists x\varphi]]_{M,A} =
      \begin{cases}
        \mathrm{True},\ \text{если найдётся $u \in \Omega \colon
        [[\varphi]]_{M,A[x \mapsto u]} = \mathrm{True}$}\\
        \mathrm{False},\ \text{иначе}
      \end{cases}$
  \end{itemize}
\end{definition}

\begin{definition}[Связность/свободность переменных]
  При <<навешивании>> квантора на формулу $Qx[\varphi]$ переменная $x$ в
  формуле $\varphi$ становится \textit{связной}. Переменные, которые не
  являются связными, называются \textit{свободными}. Одна и та же
  переменная может быть в одном месте свободной, а в другом месте -- связной.

  Пример:
  \[
    (\forall x[P(x, y) \im x = f(x, y)] \land \exists y[x = y])
  \]
  Связные переменные --- $x$ в первом кванторе, $y$ во втором кванторе.
\end{definition}

\begin{definition}
  Формула $\varphi$ называется \textit{истинной в модели $M$}, если
  для любой оценки $A$
  \[
    [[\varphi]]_{M,A} = \mathrm{True}
  \]
  Обозначение: $M \models \varphi$
\end{definition}

\begin{definition}
  Формула $\varphi$ называется \textit{общезначимой}, если она
  истинна в любой модели. Обозначение: $\models \varphi$.
\end{definition}

\begin{definition}
  Формулы $\varphi_1, \varphi_2$ называются \textit{эквивалентными},
  если для любой модели $M$ и любой оценки $A$ верно:
  \[
    [[\varphi_1]]_{M,A} = [[\varphi_2]]_{M,A}
  \]
  Обозначение: $\varphi_1 \eq \varphi_2$.
\end{definition}

\begin{lemma}[Сведение эквивалентности к общезначимости]
  \[
    \varphi_1 \eq \varphi_2 \iff \models \varphi_1 \leftrightarrow \varphi_2
  \]
\end{lemma}

Полезные эквивалентности:
\begin{align*}
  \exists x[\varphi(x)] &\eq \exists y [\varphi(y)]\\
  \forall x [\varphi(x)] &\eq \forall y [\varphi(y)]\\
  \n \exists x [\varphi] &\eq \forall x [\n \varphi]\\
  \n \forall x [\varphi] &\eq \exists x [\n \varphi]\\
  \exists x [\varphi] \land \psi &\eq \exists x [\varphi \land
  \psi]\ \text{если $\psi$ не содержит $x$}\\
  \forall x [\varphi] \land \psi &\eq \forall x [\varphi \land
  \psi]\ \text{если $\psi$ не содержит $x$}\\
  &\text{аналогично для $\lor$}
\end{align*}

\begin{definition}
  Формула имеет предваренную нормальную форму, если все предикаты
  вынесены <<наружу>>.
\end{definition}

\begin{theorem}
  У любой формулы есть эквивалентная ей формула в предваренной нормальной форме.
\end{theorem}

\subsection{Выразимые предикаты}

Пусть $M = (\Omega, I)$ --- некоторая модель. Рассмотрим $k$
переменных $x_1, \dots, x_n$ и формулу $\varphi$ такую, что её
множество свободных переменных содержится в множестве $\set{x_1,
\dots, x_n}$. Тогда можно считать, что значение формулы $\varphi$
зависит от $k$ параметров. При этом формуле $\varphi$ будет
соответствовать функция $\Omega^k \to \set{\mathrm{False},
\mathrm{True}}$. Эквивалентным образом можно считать, что это не
функция, а предикат $P_{\varphi}$ арности $k$;
\[
  (u_1, \dots, u_k) \in P_{\varphi} \iff [[\varphi]]_{M,A[x_i
  \mapsto u_i]} = \mathrm{True}
\]
Пример 1: $(\NN, 0, s, +, \m, =)$ --- модель. Рассмотрим формулу
\[
  \varphi = \exists z [x_1 + z = x_2]
\]
Свободные переменные --- $x_1, x_2$. Тогда задаётся предикат арности 2
\[
  P_{\varphi} = \set{(u_1, u_2) \mid u_1 \leqslant u_2}
\]
Пример 2: модель та же, формула
\[
  \psi = \exists z \exists y [\n z = 0 \land \n z = s(0) \land
  \n y = 0 \land \n y = s(0) \land x = yz]
\]
Тогда задаётся следующий предикат:
\[
  P_{\psi} = \set{n \mid n \text{ --- составное} \land n \geqslant 2}
\]

\begin{definition}[Выразимые предикаты]
  Предикат $R$ на множестве $\Omega$ называется \textit{выразимым} в
  модели $M$, если существует формула $\varphi$ такая, что $R = P_{\varphi}$.

  Подмножество $\Omega$ по сути является предикатом арности 1.
  Поэтому мы будем также говорить и о выразимых множествах.
\end{definition}

\subsection{Изоморфизмы}

\begin{definition}[Изоморфизм]
  Пусть $M_1 = (\Omega_1, I_1)$ и $M_2 = (\Omega_2, I_2)$ --- две
  модели такие, что интерпретации интерпретируют одни и те же имена и
  арности этих имён совпадают. Функция $\alpha \colon \Omega_1 \to
  \Omega_2$ называется \textit{изоморфизмом}, если она биекция и
  выполнены следующие условия:
  \begin{enumerate}
    \item $\alpha(I_1[c]) = I_2[c]$ для каждого константного имени из $I_1$;

    \item $\alpha(I_1[f](u_1, \dots, u_k)) = I_2[f](\alpha(u_1),
      \dots, \alpha(u_k))$ для каждого функционального имени $f$ из
      $I_1$ и $(u_1, \dots, u_k) \in \Omega_1^k$;

    \item $(u_1, \dots, u_k) \in I_1[R] \iff (\alpha(u_1), \dots,
      \alpha(u_k)) \in I_2[R]$ для каждого предикатного имени $R$ из
      $I_1$ и $(u_1, \dots, u_k) \in \Omega_1^k$.
  \end{enumerate}
\end{definition}

\begin{theorem}
  Пусть $M_1 = (\Omega_1, I_1)$ и $M_2 = (\Omega_2, I_2)$ и $\alpha$
  --- изоморфизм, тогда для лююбого терма $t$ и любой формулы
  $\varphi$ верно, что (свободные) переменные, которые содержатся во
  множестве $\set{x_1, \dots, x_k}$ верно следующее:
  \begin{enumerate}
    \item $\alpha([[t]]_{M_1, A[x_i \mapsto u_i]}) = [[t]]_{M_2,A[x_i
      \mapsto \alpha(u_i)]}$

    \item $[[\varphi]]_{M_1, A[x_i \mapsto u_i]} = [[\varphi]]_{M_2,A[x_i
      \mapsto \alpha(u_i)]}$
  \end{enumerate}
  \begin{proof}
    \begin{enumerate}
      \item
        Докажем по индукции.

        \textbf{База}:
        \begin{itemize}
          \item
            \[
              \alpha([[c]]_{M_2,A[\dots]}) = \alpha(I_1[c]) = I_2[c] =
              [[c]]_{M_2,A[\dots]}
            \]

          \item
            \[
              \alpha([[x_i]]_{M_1,A[x_i \mapsto u_i]}) = \alpha(u_i) =
              [[x_i]]_{M_2,A[x_i \mapsto u_i]}
            \]
        \end{itemize}

        \textbf{Шаг}:
        \begin{align*}
          \alpha([[f(t_1, \dots, t_n)]]_{M_1,A[\dots]}) &=
          \alpha(I_1[f]([[t_1]]_{M_1,A}, \dots, [[t_n]]_{M_1,A})) =\\&=
          I_2[f](\alpha([[t_1]]_{M_1,A}), \dots, \alpha([[t_n]]_{M_1,A})) =\\&=
          I_2[f]([[t_1]]_{M_2,A[\dots]}, \dots, [[t_n]]_{M_2,A[\dots]}) =
          [[f(t_1, \dots, t_n)]]
        \end{align*}

      \item
        \begin{itemize}
          \item  Знаем, что
            \[
              [[t_1 = t_2]]_{M_1,A_1} \iff [[t_1]]_{M_1,A_1} = [[t_2]]_{M_1,A_1}
            \]
            Тогда, используя инъективность функции $\alpha$ получим, что
            это эквивалентно
            \[
              \underbrace{\alpha([[t_1]]_{M_1,A_1})}_{[[t_1]]_{M_2,A_2}}
              =
              \underbrace{\alpha([[t_2]]_{M_1,A_1})}_{[[t_2]]_{M_2,A_2}}
              \iff [[t_1 = t_2]]_{M_2,A_2} = \mathrm{True}
            \]

          \item
            \begin{align*}
              [[R(t_1, \dots, t_n)]]_{M_1,A_1} = \mathrm{True} &\iff
              ([[t_1]]_{M_1,A_1}, \dots, [[t_n]]_{M_1,A_1}) \in
              I_1(R) \iff\\&\iff
              (\alpha([[t_1]]_{M_1,A_1}), \dots,
              \alpha([[t_n]]_{M_1,A_1})) \in I_2(R) \iff\\&\iff
              ([[t_1]]_{M_2,A_2}, \dots, [[t_n]]_{M_2,A_2}) \in
              I_2(R) \iff [[R(t_1, \dots, t_n)]]_{M_2,A_2}
            \end{align*}

          \item Для бинарных функций доказательства аналогичны,
            докажем для конъюнкции:
            \begin{align*}
              [[\varphi_1 \land \varphi_2]]_{M_1,A_1} &=
              [[\varphi_1]]_{M_1,A_1} \land [[\varphi_2]]_{M_1,A_1} =\\&=
              [[\varphi_1]]_{M_2,A_2} \land [[\varphi_2]]_{M_2,A_2} =\\&=
              [[\varphi_1 \land \varphi_2]]_{M_2,A_2}
            \end{align*}

          \item
            \[
              [[\forall y [\varphi]]]_{M_1,A_1} = \mathrm{True} \iff
              \forall u \in \Omega_1 ([[\varphi]]_{M_1,A_1[y \mapsto
              u]} = \mathrm{True})
            \]
            \[
              [[\forall y [\varphi]]]_{M_2,A_2} = \mathrm{True} \iff
              \forall u' \in \Omega_2 ([[\varphi]]_{M_2,A_2[y \mapsto
              u']} = \mathrm{True})
            \]
            $\impliedby$: Пусть $[[\varphi]]_{M_1,A[x_i \mapsto u_i,
            y \mapsto u]} = \mathrm{False} \iff [[\varphi]]_{M_2A[x_i
            \mapsto \alpha(u_i), y \mapsto \alpha(u) = u']} = \mathrm{False}$

            $\implies$:
        \end{itemize}
    \end{enumerate}
    TODO: дописать
  \end{proof}
\end{theorem}

\begin{definition}
  \textit{Автоморфизмом} называется изоморфизм в себя, т.е. функция
  $\alpha \colon \Omega \to \Omega$, являющаяся изоморфизмом для
  модели $M = (\Omega, I)$.

  Очевидно, что тождественное отображение является автоморфизмом.
  Такой автоморфизм будем называть \textit{тривиальным}. Но
  существуют и нетривиальные автоморфизмы.
\end{definition}

\begin{definition}
  Предикат $R \sus \Omega^k$ \textit{сохраняется при отображении
  $\alpha \colon \Omega \to \Omega$}, если $(u_1, \dots, u_k) \in R
  \iff (\alpha(u_1), \dots, \alpha(u_k)) \in R$.
\end{definition}

\begin{theorem}[Критерий невыразимости]
  Если предикат $R$ не сохраняется при автоморфизме, то он невыразим
  никакой формулой.
  \begin{proof}
    Рассмотрим предикат $R \sus \Omega^k$, тогда $\exists \varphi
    \colon P_{\varphi} = R$. Возьмём $(u_1, \dots, u_k)$. Тогда
    \begin{align*}
      (u_1, \dots, u_k) \in R &\iff
      [[\varphi]]_{M,A[x_i \mapsto u_i]} = \mathrm{True} \iff\\&\iff
      [[\varphi_1]]_{M,A[x_i \mapsto \alpha(u_i)]} = \mathrm{True} \iff\\&\iff
      (\alpha(u_1), \dots, \alpha(u_k)) \in R
    \end{align*}
  \end{proof}
\end{theorem}

\pagebreak

\section{Лекция 11}

В пропозициональной логике мы смогли свести все высказывания к
высказываниям лишь с операторами $\n$ и $\im$. То же самое можно
сделать и в логике предикатов; в логике предикатов можно избавиться
от равенства и функциональных имён.

Рассмотрим две модели:
\[
  M_1 = (\NN; 0, s, +, \m, =),\ M_2 = (\NN; 0, S, P, M, =)
\]
Пусть для этих моделей верно следующее:
\begin{align*}
  x = s(y) &\iff (x, y) \in S\\
  x = y + z &\iff (x, y, z) \in P\\
  x = y \m z &\iff (x, y, z) \in M
\end{align*}
Возьмём формулу
\[
  \exists y [\exists t [x = \underbrace{y \m y}_{z1} + \underbrace{t
  \m t}_{z2}]]
\]
Тогда её также можно переписать, не используя функции и равенство:
\begin{align*}
  &\exists y \exists t [\exists z1 [\underbrace{y \m y = z1}_{M(z1, y,
      y)} \land \exists z2 [\underbrace{z2 = t \m t}_{M(z2, t, t)}
        \land \exists z3 [\underbrace{z3 = z1 + z2}_{P(z3, z1, z2)} \land x
  = z3]]]] \eq\\&\eq
  \exists y \exists t [\exists z1 [M(z1, y, y) \land \exists z2
  [M(z2, t, t) \land \exists z3 [P(z3, z1, z2) \land x = z3]]]]
\end{align*}

Идея состоит в том, что каждую функцию $f$ арности $k$ заменяют на
предикат $F$ арности $k + 1$, такой что
\[
  y = f(x_1, \dots, x_k) \iff (y, x_1, \dots, x_k) \in F
\]
При этом будем использовать \textit{каноническое представление}
предикатных имён, соответствующих функциональным. Для этого будем
заменять первую букву в имени функции на заглавную.

Осталось убедиться, что всё, что можно было выразить в старой модели,
можно будет выразить в новой и наоборот.

\begin{definition}
  Для модели $M$ с функциями будем называть модель $M'$
  \textit{бесфункциональным аналогом} модели $M$, если в $M'$ удалены
  все функции и добавлены предикаты, соответствующие функциям в $M$, и
  они имеют канонические имена.
\end{definition}

\begin{definition}
  Пусть $\varphi$ --- формула, которая может содержать функции. Будем
  говорить, что формула $\varphi'$ является \textit{бесфункциональным
  аналогом} формулы $\varphi$, если
  \begin{itemize}
    \item $\varphi'$ содержит те же предикаты, что и $\varphi$, при
      этом $\varphi'$ будет содержать ещё предикаты, соответствующие
      функциям из $\varphi$, переименованным каноническим образом;

    \item TODO
  \end{itemize}
\end{definition}

Хотелось бы, чтобы для новой формулы выполнялось и обратное
утверждение, т.е., если $\varphi'$ истинна в некоторой модели $M'$,
то найдётся модель $M$, в которой истинна $\varphi$ и $M'$ является
бесфункциональным аналогом $M$.

Действительно, если $\varphi'$ истинна в некоторой модели, то нет
гарантии, что предикат $F$, соответствующий функции $f$ в исходной
формуле, можно представить в виде $y = f(x_1, \dots)$. Чтобы это
всегда можно было сделать, нужно добавить условия того, что все
предикаты, сооветствующие функциям, должны обладать тотальностью и
функциональностью.

\begin{theorem}
  Для любого множества $\Gamma$ формул (которые могут содержать
  функции) существует множество формул $\Gamma'$ без функций,
  получаемое из $\Gamma$ с помощью эффективной синтаксической
  процедуры, такое, что $\Gamma$ имеет модель тогда и только тогда,
  когда $\Gamma'$ имеет модель. Более того, имея модель $\Gamma'$,
  существует простой и естественный способ преобразовать её в модель
  $\Gamma$ и наоборот.
\end{theorem}

Вообще говоря, равенство ведёт себя как предикат, но имеет некоторые
особенности. Заменим его на предикат $SAME$, т.е. везде вместо $t_1 =
t_2$ будем писать $SAME(t_1, t_2)$. При этом, если предикат $SAME$
будет интерпретироваться равенством,
\[
  (x, y) \in SAME \iff x = y,
\]
то истинность после замены будет сохраняться.

Как и в случае с функциями, хотелось бы, чтобы выполнялось и
обратное, т.е. любую модель новой формулы можно было получить
естественным преобразованием из некоторой модели старой формулы. Для
этого надо добавить свойства, которым удовлетворяют равенства,
переписанные для $SAME$.

Аксиомы равенства для $SAME$:
\begin{enumerate}
  \item $SAME$ --- отношение эквивалентности:
    \[
      \forall x [SAME(x, x)]
    \]
    \[
      \forall x [\forall y [SAME(x, y) \im SAME(y, x)]]
    \]
    \[
      \forall x [\forall y [\forall z [((SAME(x, y) \land SAME(y, z))
      \im SAME(x, z))]]]
    \]

  \item Все предикаты и функции <<уважают>> $SAME$. Для каждой
    функции $f$ арности $k$ в модели надо добавить:
    \[
      \forall x_1 \dots \forall x_k \forall y_1 \dots \forall
      y_k[(SAME(x_1, y_1) \land \dots, \land SAME(x_k, y_k)) \im
      SAME(f(x_1, \dots, x_k), f(y_1, \dots, y_k))]
    \]
    Для каждого предиката $R$ арности $k$:
    \[
      \forall x_1 \dots \forall x_k \forall y_1 \dots \forall y_k
      [(SAME(x_1, y_1) \land \dots SAME(x_k, y_k)) \im (R(x_1, \dots,
      x_k) \im R(y_1, \dots, y_k))]
    \]
\end{enumerate}

\begin{theorem}
  Пусть $M'$ --- модель формулы $\varphi_s$, тогда существует модель
  $M$, которая является моделью $\varphi$ и наоборот.
  \begin{proof}
    Пусть $M' = (\Omega', I')$ --- модель $\varphi'_s$. Возьмём $M =
    (\Omega, I)$, определённую следующим образом:
    \begin{itemize}
      \item  $\Omega = \Omega' / SAME = \set{[m] \mid m \in
        \Omega'}$, (фактор-множество), где $[m] = \set{m' \mid (m,
        m') \in SAME}$.

      \item
        \begin{itemize}
          \item $I(c) = [I'(c)]$

          \item $I(f) = f_M([m_1], \dots, [m_k]) = [f'_M(m_1, \dots, m_k)]$

          \item $I(R) = R_M \colon ([m_1], \dots, [m_k]) \in R_M \iff
            (m_1, \dots, m_k) \in R_{M'}$
        \end{itemize}
    \end{itemize}
    Остаётся проверить корректность определения.
    \[
      SAME(m_1, m_1') \land \dots \land SAME(m_k, m'_k) \implies
      SAME(f_{M'}(m_1, \dots, m_k), f_{M'}(m'_1, \dots, m'_k))
    \]
    Теперь покажем, что для любого терма $t$ верно $[[t]]_{M,A} =
    [[[t]]_{M',A'}]$, где $A(x) = [A'(x)]$. Доказывается индукцией по
    длине терма.

    Для любой формулы $\psi$, использующей те же функции и предикаты,
    что $\varphi$, верно $[[\psi]]_{M,A} = [[\psi]]_{M',A'}$.
    Доказывается индукцией по длине формулы.
  \end{proof}
\end{theorem}

\pagebreak

\section{Лекция 12 TODO}

\pagebreak

\section{Лекция 13}

\subsection{Выводы}

\begin{definition}
  \textit{Выводом} в исчислении предикатов называется
  последовательность строк, каждая строка при этом может быть:
  \begin{enumerate}
    \item \textit{Тавтологией}, которые представляют собой булевы
      формулы-тавтологии, в которых переменные заменены на формулы
      логики предикатов;

    \item \textit{Аксиомой}, которые будут задаваться схемами. Список
      схем может отличаться.
      \[
        (\forall x [\varphi(x)] \im \varphi(t)),\ \text{где $\varphi$
        --- формула, $x$ --- переменная, $t$ --- терм}
      \]

    \item Или получены из предыдущих строк с помощью \textit{правила
      вывода}. Правил вывода с непустым множеством посылок у нас будет
      два: \textit{Modus Ponens (MP)}:
      \[
        \frac{\varphi, (\varphi \im \psi)}{\psi}
      \]
      и \textit{правило обобщения (Gen)}:
      \[
        \frac{\varphi}{\forall x [\varphi]}.
      \]
  \end{enumerate}
\end{definition}

\begin{remark}
  Мы знаем теорему о полноте для логики высказываний,
  поэтому мы могли бы вместо тавтологий добавить схемы аксиом Гильберта
  (заменяя переменные на формулы). Множество выводимых формул при этом
  не изменяется, но усложняются выводы.
\end{remark}

\begin{definition}
  Схемы мы будем представлять в виде обычных формул, но с другим
  пониманием имён:
  \begin{center}
    \begin{tabular}{c|c}
      В формуле & В схеме \\\hline
      имя константы & терм \\\hline
      имя переменной & переменная \\\hline
      имя предиката & формула
    \end{tabular}
  \end{center}
\end{definition}

Заметим, что тут нам пригодится то, что мы разрешили иметь нульарные
предикатные имена в формулах.

\subsection{Ограничения на подстановки}

Хотелось бы, чтобы после вставки смысл формулы не меняелся. В
частности, подстановка не должна портить истинность.

Рассмотрим следующую схему:
\[
  \forall x [\varphi(x)] \im \varphi(c).
\]
Она очевидно истинна при любой разумной интерпретации. Теперь вместо
$\varphi(x)$ подставим $\exists y [y = x]$, а вместо $c$ подставим $y
+ 1$, получится
\[
  \forall x [\exists y [x = y]] \im \exists y [y + 1 = y].
\]
Это неверное утверждение, например, в $\NN$.

Нужно вводить ограничения, а именно:
\begin{enumerate}
  \item Термы не должны содержать переменные, такие что терм попадёт
    в область действия квантора по этой переменной;

  \item Не должно быть кванторов по переменным, которые попадают в
    область действия кванторов по тем же переменным.
\end{enumerate}

\begin{definition}[Общезначимость]
  Формула $\varphi$ называется \textit{общезначимой}, если она
  истинна во всех моделях при всех оценках, в которых
  интерпретируются все имена, используемые в этой формуле.

  Схема называется \textit{общезначимой}, если все корректные
  подстановочные варианты этой схемы являются общезначимыми.
\end{definition}

Доказательство общезначимости
\[
  \forall x [\varphi(x)] \im \varphi(t).
\]
\begin{proof}
  $\models_{M, A} \forall x [\varphi(x)] \iff \forall m \in M \colon
  \models_{M, A[x \mapsto m]} \varphi(x) \implies \models_{M, A[x
  \mapsto m']} \varphi(x) \iff \models_{M,A} \varphi(c)$, где $[[t]]_{M,A} = m'$
\end{proof}

\subsection{Схемы аксиом}

\begin{itemize}
  \item Универсальная конкретизация (UI):
    \[
      (\forall x [\varphi(x)] \im \varphi(c));
    \]

  \item Экзистенциальное введение (EI):
    \[
      (\varphi(c) \im \exists x [\varphi(x)]);
    \]

  \item Универсальное упрощение (US):
    \[
      (\forall x [(\psi \im \varphi(x))] \im (\psi \im \forall x
      [\varphi(x)])),\ \text{если $\psi$ не содержит свободной переменной $x$};
    \]

  \item Экзистенциальное упрощение (ES):
    \[
      (\forall x [(\varphi(x) \im \psi)] \im (\exists x [\varphi(x)]
      \im \psi)),\ \text{если $\psi$ не содержит свободной переменной $x$}
    \]
\end{itemize}

\begin{theorem}[О корректности]
  Все схемы аксиом выше корректны, т.е. при правильной подстановке
  дают общезначимые формулы, а правила вывода (MP) и (GEN) сохраняют
  общезначимость.
  \begin{proof}
    (UI) проверяли, доказательство (EI) аналогичное. Проверка (MP)
    аналогична доказательству для логики высказываний. Докажем
    утверждение для (Gen):
    \[
      \frac{\varphi(x)}{\forall x [\varphi(x)]}.
    \]
    По сути надо доказать следующие две леммы:
    \begin{lemma}
      Пусть $M$ --- модель, $A$ --- оценка, для любых формул
      $\varphi, \psi$ верно
      \[
        (\models_{M,A} (\varphi \im \psi)) \land (\models_{M,A}
        \varphi) \implies \models_{M,A} \psi
      \]
    \end{lemma}
    \begin{lemma}
      Пусть $M$ --- модель, $\varphi(x)$ --- формула. Тогда, если для
      любой оценки $A$ истинно $\models_{M,A} \varphi(x)$, то для
      любой оценки $A$ истинно $\models_{M,A} \forall x [\varphi(x)]$.
    \end{lemma}
    TODO?
  \end{proof}
\end{theorem}

\end{document}
